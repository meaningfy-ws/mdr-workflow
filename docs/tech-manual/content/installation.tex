\section{Requirements}
\label{sec:requirements}

	RedHat.
	
	The inventory of pre-configured ports is provided in Table \ref{tab:port-inventory}.

	\begin{longtable}[c]{@{}p{3.64cm}p{1.25cm}p{1.25cm}p{1.9cm}p{5cm}@{}}
		\toprule
		Service name & HTTP port UI & HTTP port API & FTP port & Mounted volume \\* \midrule
		\endfirsthead
		%
		\multicolumn{5}{c}%
		{{\bfseries Table \thetable\ continued from previous page}} \\
		\endhead
		%
		\bottomrule
		\endfoot
		%
		\endlastfoot
		%
		RDF differ & 8030 & 4030 &  &  \\* \hline
		RDF differ dedicated Fuseki &  & 3030 &  & rdf-differ-fuseki \\* \hline
		RDF validator & 8010 & 4010 &  &  \\* \hline
		RDF fingerprinter & 8020 & 4020 &  &  \\* \hline
		RDF fingerprinter dedicated Fuseki &  & 3020 &  & rdf-fingerprinter-fuseki \\* \hline
		LinkedPipes ETL - storage &  & 8063 &  & linkedpipes-logs, linkedpipes-data-storage, linkedpipes-configuration \\* \hline
		LinkedPipes ETL - executor &  & 8065 &  & linkedpipes-logs, linkedpipes-data-execution, linkedpipes-configuration \\* \hline
		LinkedPipes ETL - monitor &  & 8061 & 2221, 2222, 2225 & linkedpipes-logs, linkedpipes-data-execution, linkedpipes-configuration \\* \hline
		LinkedPipes ETL - frontend & 8060 &  &  & linkedpipes-logs, linkedpipes-configuration \\* \hline
		LinkedPipes ETL - dedicated Fuseki &  & 3060 &  & linkedpipes-fuseki \\* \hline
		Jenkins & 8080 & 50000 &  & jenkins-home \\* \hline
		Camunda BPMN engine & 8040 &  &  & rdf-camunda \\* \bottomrule
		\caption{Port usage inventory}
		\label{tab:port-inventory}\\
	\end{longtable}

%	\vfil
	The minimal hardware requirements are as follows 
	\begin{enumerate}
		\item CPU: 3.2 Ghz quad core 
		\item RAM: 16GB 
		\item SDD system: 32GB 
		\item SDD data: 128GB 
	\end{enumerate}	

\section{Installation}
\label{sec:installation}


	Change directory into the project folder to be able to use the Makefile commands to start and stop services will be available. 	

	In order to run the services it is necessary to have the project dependencies installed. 

	To do so, run the following commands:
	
	\begin{lstlisting}[language=bash,]
make install-os-dependencies
	\end{lstlisting}
	
	To start the services using Makefile

\begin{lstlisting}[language=bash,]
make start-services
\end{lstlisting}

	To stop the services using Makefile
	
\begin{lstlisting}[language=bash,]
make stop-services
\end{lstlisting}		

	To start services without Makefile first prepare the volume with LinkedPipes ETL configurations file like this
	
\begin{lstlisting}[language=bash,]
docker rm temp | true
docker volume rm linkedpipes-configuration | true
docker volume create linkedpipes-configuration
docker container create --name temp -v linkedpipes-configuration:/data busybox
docker cp ./docker/linkedpipes-etl/configuration/configuration.properties temp:/data
docker rm temp
\end{lstlisting}		

	then start the services
	
\begin{lstlisting}[language=bash,]
docker-compose --file docker/docker-compose.yml --env-file docker/.env up -d
\end{lstlisting}

	To stop the services run

\begin{lstlisting}[language=bash,]
docker-compose --file docker/docker-compose.yml --env-file docker/.env down
\end{lstlisting}


	\subsection{Set custom templates}
 
	 RDF Differ, RDF Validator and RDF Fingerprinter support custom templates. The custom templates are stored in Docker volumes, and are explicitly marked to be externally created. Therefore, before running the service for the first time, ensure the volumes exist by running the following make command:

\begin{lstlisting}[language=bash,]
make build-template-volumes
\end{lstlisting}	 
	 	 
	 Provided that the custom template is available on the host system in a \textit{location to template} you can set it by using the following makefile targets, which are prepared to facilitate this operation. 
	 
\begin{lstlisting}[language=bash,]
make location=<location to template> differ-set-report-template
make location=<location to template> validator-set-report-template
make location=<location to template> fingerprinter-set-report-template
\end{lstlisting}	 	 

	The detailed explanation on how to configure them is provided in the Configuration section for each of these services (See Section \ref{sec:rdf-differ-ct}, \ref{sec:rdf-fingerprinter-ct}, \ref{sec:rdf-validator-ct} ). 
 
