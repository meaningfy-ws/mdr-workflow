\section{Configuration}
\label{sec:configuration}
The deployment suite of micro-services is defined docker-compose.yml file. At deployment and at runtime, the service configurations are provided through OS environment variables available in the \textit{.env} file. The role of the .env file is to enable the system administrators to easily change default configurations as necessary in the context of their environment.

The suite of micro-services is built, started and shut down via docker-compose, a tool designed especially for managing multi-container Docker applications, by describing them in a single file. Then, with a single command, you create and build, start or stop all the services using that configuration file.

In order to avoid hard coding parameters, docker-compose allows you to define them externally. You have the option to define them as operating system level environment variables or provide them in a single file, which is passed as a parameter to the docker-compose tool using the \textit{--env-file} command line argument. Having them in a single file makes much more sense and it is more pragmatic, as you can see and manage all parameters in one place, add the file to the version control system (the contents of the file will evolve and be in sync with the actual code) and have different files for different environments.

The file is usually named \textit{.env} and contains all of the parameters that you want to be able to change and that you need to build and run the defined containers. 

Having the parameters in an \textit{.env} file is very useful in a multitude of scenarios, where you would want to have different configurations for different environments where you might want to deploy. As a more specific example, consider a continuous delivery pipeline and the URLs and ports you want your containers to bind (or to connect) to. You thus can easily have two \textit{.env} files, one named \textit{test.env} and one named \textit{acceptance.env}. Each file would have the same declared variables, but with different values for each of the continuous delivery pipeline stage where it’s being deployed. The benefit is that you deploy and test/use the same containers/artifacts and are able to configure them, on the spot, according to the environment that they are integrated with.


Let’s take, for example, the RDF Differ user interface Docker container, which is defined, in the \textit{docker-compose.yml} file as it follows:
\begin{lstlisting}[]
    rdf-differ-ui:
    	container_name: rdf-differ-ui
    	image: meaningfy/rdf-differ-ui:latest
    	ports:
    		- ${RDF_DIFFER_UI_PORT}:${RDF_DIFFER_UI_PORT}
    	env_file: .env
    	restart: always
    	networks:
    		- mydefault
\end{lstlisting}
The variable used in the definition of this service is just one, \textbf{\textit{RDF\_DIFFER\_UI\_PORT}}. And the place where docker-compose will look for that variable is specified in the \textbf{\textit{env\_file: .env}} line.

Now, if you look in the “.env” file, you will quickly see that the variable is defined as \textbf{\textit{RDF\_DIFFER\_UI\_PORT=8030}}. Change the value of the port, rebuild the micro-services and RDF Differ will no longer be listening on 8030, but on the new port that you specified.


This section describes the important configurations options available for each of the services.

\subsection{RDF differ}

The RDF differ application exposes an API and an UI and depends on a dedicated triple store. the RDF diff API is the core service providing the RDF diffing functionality. The URL and port are described below, as well as the request timeout:

\begin{longtable}[c]{@{}p{3.5cm}p{3.5cm}l@{}}
	\toprule
	Description                       & Value                 & Associated variable            \\* \midrule
	\endfirsthead
	%
	\multicolumn{3}{c}%
	{{\bfseries Table \thetable\ continued from previous page}}                                \\
	\endhead
	%
	\bottomrule
	\endfoot
	%
	\endlastfoot
	%
	Service URL                       & http://rdf-differ-api & RDF\_DIFFER\_API\_LOCATION     \\
	Service API port                  & 4030                  & RDF\_DIFFER\_API\_PORT         \\
	Is in debug mode                  & True                  & RDF\_DIFFER\_DEBUG             \\
	Service UI port                   & 8030                  & RDF\_DIFFER\_UI\_PORT          \\
	Web server worker process timeout & 1200                  & RDF\_DIFFER\_GUNICORN\_TIMEOUT \\* \bottomrule
	\caption{RDF differ configurations}
	\label{tab:my-table1}                                                                      \\
\end{longtable}

Please note that the domain specified in in the URL is only available inside the Docker network and is not visible from the outside. Its purpose is to provide a named way for a service to connect to another service. 

\subsection{Celery worker and Redis}
Celery is a simple, flexible, and reliable distributed system to process vast amounts of messages, while providing operations with the tools required to maintain such a system. It’s a task queue with focus on real-time processing.

In the rdf differ project it serves the purpose of enabling multiprocessing of both diff creation and report generation.

Redis is used as celery's backend to store information about tasks.
Redis is an open source (BSD licensed), in-memory data structure store, used as a database, cache, and message broker. Redis provides data structures such as strings, hashes, lists, sets, sorted sets with range queries, bitmaps, hyperloglogs, geospatial indexes, and streams.

A docker configuration is provided for both celery and redis (\texttt{celery-worker} and \texttt{redis})

The RDF differ application uses the following Celery environment variables

\begin{longtable}[c]{@{}p{3.5cm}p{3.5cm}l@{}}
	\toprule
	Description    & Value                  & Associated variable          \\* \midrule
	\endfirsthead
	%
	\multicolumn{3}{c}%
	{{\bfseries Table \thetable\ continued from previous page}}            \\
	\endhead
	%
	\bottomrule
	\endfoot
	%
	\endlastfoot
	%
	Redis location & \texttt{redis://redis} & RDF\_DIFFER\_REDIS\_LOCATION \\
	Redis port     & 6379                   & RDF\_DIFFER\_REDIS\_PORT     \\*\bottomrule
	\caption{Celery environment configurations}
	\label{tab:celery-env}                                                 \\
\end{longtable}
\subsection{RDF differ dedicated triple store}

RDF differ depends on a Fuseki triple store to calculate and persist the diffs. The available configurations are described below. 

\begin{longtable}[c]{@{}p{4cm}p{2cm}l@{}}
	\toprule
	Description                        & Value                    & Associated variable                  \\* \midrule
	\endfirsthead
	%
	\multicolumn{3}{c}%
	{{\bfseries Table \thetable\ continued from previous page}}                                          \\
	\endhead
	%
	\bottomrule
	\endfoot
	%
	\endlastfoot
	%
	Admin account password             & admin                    & RDF\_DIFFER\_FUSEKI\_ADMIN\_PASSWORD \\
	User name                          & admin                    & RDF\_DIFFER\_FUSEKI\_USERNAME        \\
	Password                           & admin                    & RDF\_DIFFER\_FUSEKI\_PASSWORD        \\
	Folder where Fuseki stores data    & ./data/diff              & RDF\_DIFFER\_FUSEKI\_DATA\_FOLDER    \\
	External port                      & 3030                     & RDF\_DIFFER\_FUSEKI\_PORT            \\
	Internal port                      & 3030                     &                                      \\
	Additional arguments passed to JVM & -Xmx2g                   & RDF\_DIFFER\_FUSEKI\_JVM\_ARGS       \\
	URL                                & http://rdf-differ-fuseki & RDF\_DIFFER\_FUSEKI\_LOCATION        \\* \bottomrule
	\caption{RDF differ dedicated triple store configurations}
	\label{tab:my-table2}                                                                                \\
\end{longtable}
