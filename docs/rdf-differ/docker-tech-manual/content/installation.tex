\section{Requirements}
\label{sec:requirements}

Although Docker can be executed on any platform, for performance and security reasons we recommend using a Linux OS with kernel version 5.4x or higher. The services have been tested on Ubuntu 20 server. 

There is a range of ports that must be available on the host machine as they will be bound to by different docker services. Although the system administrator may choose to change them by changing the values in of specific environment variables. The inventory of pre-configured ports is provided in Table \ref{tab:port-inventory}.

\begin{longtable}[c]{@{}p{3.64cm}p{1.25cm}p{1.25cm}p{1.9cm}p{5cm}@{}}
	\toprule
	Service name     & HTTP port UI & HTTP port API             \\* \midrule
	\endfirsthead
	%
	\multicolumn{5}{c}%
	{{\bfseries Table \thetable\ continued from previous page}} \\
	\endhead
	%
	\bottomrule
	\endfoot
	%
	\endlastfoot
	%
	RDF differ       & 8030         & 4030                      \\* \hline
	dedicated Fuseki &              & 3030                      \\* \hline
	Redis            &              & 6379                      \\* \bottomrule
	\caption{Port usage inventory}
	\label{tab:port-inventory}                                  \\
\end{longtable}

%	\vfil
The minimal hardware requirements are as follows 
\begin{enumerate}
	\item CPU: 3.2 Ghz quad core
	\item RAM: 16GB
	\item SDD system: 32GB
	\item SDD data: 128GB
\end{enumerate}

\section{Installation}
\label{sec:installation}

In order to run the services it is necessary to have Docker server and docker-compose tool installed. To install them following the instructions provided at the following locations

\begin{enumerate}
	\item Docker - \url{https://docs.docker.com/engine/install}
	\item Docker Compose - \url{https://docs.docker.com/compose/install}
\end{enumerate}

In case you are using Debian-like OS such as Ubuntu, you may simply run the following Bash commands to install and set the appropriate permissions. 

\begin{lstlisting}[language=bash,]
sudo apt -y install docker.io docker-compose git make
sudo groupadd docker
sudo usermod -aG docker $USER
newgrp docker
	\end{lstlisting}

Next, copy the rdf differ zip on the system you intend to run it and unzip it.

Then change directory into the \textit{rdf-differ} folder and Makefile commands to start and stop services will be available. 	

Downloading the Docker images will be triggered automatically on first request to start the services. 

To start the services using Makefile

\begin{lstlisting}[language=bash,]
make start-services
\end{lstlisting}

To stop the services using Makefile

\begin{lstlisting}[language=bash,]
make stop-services
\end{lstlisting}

To start services without Makefile:

\begin{lstlisting}[language=bash,]
docker-compose --file docker/docker-compose.yml --env-file docker/.env up -d
\end{lstlisting}

To stop the services run

\begin{lstlisting}[language=bash,]
docker-compose --file docker/docker-compose.yml --env-file docker/.env down
\end{lstlisting}

