\section{Add a new application profile template}

There are several application profiles already provided within the system that resides in \texttt{resource\slash templates} folder. For adding a new application profile create a new folder under \texttt{resource\slash templates} with the name of your new application profile and following the structure explained below.

Folder structure needed for adding a new application profile:

~
~
~

\begin{lstlisting}
templates/
  diff_report/
  new_application_profile/
   queries/      <--- contains SPARQL queries
    query1.rq
    query2.rq
   template_variants/
    html/        <--- contains files for a html template
    json/        <--- contains files for a json template
\end{lstlisting}

\subsection{HTML template variant}
\subsubsection{Folder structure}

\begin{lstlisting}
html/
  config.json   <--- configuration file
  templates/    <--- jinja html templates
    file1.html
    file2.html
    main.html
\end{lstlisting}

\textit{Note:} make sure that in the templates folder there is a file named the same as the one defined in the \textit{config.json} file (i.e \texttt{"template": "main.html"})

\subsubsection{Template structure}

The HTML template is built be combining four major parts as layout, main, macros and sections. The layout file \textit{layout.html} will have the rules of how the report will look like in terms of positioning and styling. Macros will contain all the jinja2 macros used across the template. A section represents the result of a query that was run with additional html and will be used to build the report. As the name suggest the main file of the html template is main.html. Here is where every other file that are a different section in the report are included and will form the HTML report.

Example of including a section in the main html file:
\begin{lstlisting}

\end{lstlisting}

Each section file has one or more variables where the SPARQL query result is saved as a pandas dataframe.

Example:
\begin{lstlisting}
)
\end{lstlisting}

\textit{Note:} the system has in place an autodiscover process for the SPARQL queries in the queries folder. Make sure that the file name added for the variable above (\textit{added\_instance\_concept.rq}) exists in the queries folder.

\subsubsection{Adjusting an existing template}
\textit{Adding a new query/section}

To add a query a new file needs to be created and added into the queries folder as the system will autodiscover this. After this is done a new html file that will represent a new section needs to be created. The content of this is similar to the existing ones and the only thing that needs to be adjusted will be the query file name in the content variable definition as presented below:

\begin{lstlisting}
)
\end{lstlisting}

As a final step, the created html file needs to be included in the report and to do this it has to be included in the main.html file by using the include block.

\begin{lstlisting}
--- relative path to the new html path

\end{lstlisting}

For adding a count query that will be used in the statistics section the steps are a bit different. First, will need to add the new query file following the naming conventions and adding the prefix \texttt{count\_} to the file name in queries folder. After this, the statistics.html will need to be modified as follows:

\begin{enumerate}
  \item Create a new row in the existing table by using \texttt{<tr>} tag.
  \item Create the necessary columns for the newly created row. Each row should have 7 values as this is the defined table structure \textit{(Property group, Property, Added, Deleted, Updated, Moved, Changed)} and each of them should be included by using a \texttt{<td>} tag if you are not using the block below to autogenerate this.
\end{enumerate}

\begin{lstlisting}

  

  {{ mc.count_value(content) }}

\end{lstlisting}

\textit{Note:} the order of the cells is important. If you don't want to include a type of operation just create a \texttt{<td>} with a desired value (i.e \texttt{<td>N/A</td>}). To avoid confusions, count queries should be added for all type of operations. The example below will show how to add a complete row in the statistics section of the report:

\begin{lstlisting}
<tr>
  <td>Name of the property group</td>
  <td>Name of the property</td>
  --- this will bring the number generated from the SPARQL query for added occurences and will create the <td> tag
      
      
      

      {{ mc.count_value(content) }}
      
      
  --- this will bring the number generated from the SPARQL query for deleted occurences and will create the <td> tag
      
      
      

      {{ mc.count_value(content) }}
      
      
  --- this will bring the number generated from the SPARQL query for updated occurences and will create the <td> tag
      
      
      

      {{ mc.count_value(content) }}
      
      
  --- this will bring the number generated from the SPARQL query for moved occurences and will create the <td> tag
      
      
      

      {{ mc.count_value(content) }}
      
      
  --- this will bring the number generated from the SPARQL query for changed occurences and will create the <td> tag
      
      
      

      {{ mc.count_value(content) }}
      

</tr>
\end{lstlisting}

\textit{Removing a query/section}

To remove a section from the existing report you just need to delete or comment the include statement from the main.html file. If you decide to delete the include statement it's recommended to delete the query from the queries folder as well to avoid confusions later on.

~
~

Example of include statement to remove:
\begin{lstlisting}

\end{lstlisting}

~

To remove a row from the statistics section just delete or comment the \texttt{<tr>} block from the \textit{statistics.html} file:
\begin{lstlisting}
<tr>
  <td>Labels</td>
  <td>skos:prefLabel</td>
  
      
      
      

      {{ mc.count_value(content) }}
      
      
  
      
      
      

      {{ mc.count_value(content) }}
      
      
  
      
      
      

      {{ mc.count_value(content) }}
      
      
  
      
      
      

      {{ mc.count_value(content) }}
      
      
  
      
      
      

      {{ mc.count_value(content) }}
      
</tr>
\end{lstlisting}

\subsection{JSON template variant}
\subsubsection{Folder structure}

\begin{lstlisting}
json/
  config.json   <--- configuration file
  templates/    <--- jinja json templates
    main.json
\end{lstlisting}

\textit{Note:} make sure that in the templates folder there is a file named the same as the one defined in the \textit{config.json} file (i.e \texttt{"template": "main.json"})

\subsubsection{Template structure}
The JSON report is automatically built by running all queries that are found in the \textit{queries} folder as the system has autodiscover process for this. In the beginning of this report there will be 3 keys that will show the metadata of the report like dataset used, created time and application profile used. Each query result can be identified in the report by the filename and will contain a results key that will represent the result set brought back by the query:

\begin{lstlisting}
{
  --- Metadata
  
  "dataset_name": "name of dataset",
  "timestamp": "time",
  "application_profile": "application profile namme",
  
  --- Query result set
  
  "count_changed_property_concept_definition.rq":
  {
      "head":
      {
          "vars":
          [
              "entries"
          ]
      },
      "results":
      {
          "bindings":
          [
              {
                  "entries":
                  {
                      "datatype": "http://www.w3.org/2001/XMLSchema#integer",
                      "type": "literal",
                      "value": "0"
                  }
              }
          ]
      }
  }
}
\end{lstlisting}

\subsubsection{Adjusting an existing template}
\textit{Removing a query/section}
To remove a query result set from the report simply remove the query from the queries folder.

\textit{Note:} doing this will also affect the HTML template and it's recommended to adjust the html template, if this exists as a template variant for the application profile that you are working with, following the instruction above to avoid errors when generating the HTML template variant.