\section{Add a new application profile}

There is an application profile already provided within the system that resides in \texttt{resource\slash aps} folder called \texttt{main} which contain the following shape files: \texttt{skosShapes.shapes.ttl}, \texttt{euvocShapes.shapes.ttl}, and \texttt{extensionShapes.shapes.ttl}. For adding a new application profile create a new folder under \texttt{resource\slash aps} with the name of your new application profile and add the shape files inside. It then should become available to the system; this can be checked by calling the API Application Profiles endpoint described here in chapter \ref{api:get-aps}.

Folder structure of application profiles:

\begin{lstlisting}
resources/
  aps/                <--- contains application profiles
    main/             <--- provided by default
        euvocShapes.shapes.ttl
        extensionShapes.shapes.ttl
        skosShapes.shapes.ttl
    alternative/      <--- example of another application profile
        other.shapes.ttl
        alternative.shapes.ttl
\end{lstlisting}


\section{Custom templates}

To configure the templates used for generating the custom validation reports you have to modify the currently available ones, found at \texttt{resource\slash templates}. 

The templates are written in \textit{Jinja2} templating language \citep{jinja2}. The data source access is facilitated through the \textit{eds4jinja2} library \citep{eds4jinja2}. If you are familiar with Jinja2 language a short introduction to how to use eds4jinja2 is available on the documentation page\footnote{\url{https://eds4jinja2.readthedocs.io/en/latest/}}. Also the default template can be seen as an example accessible in the repository\footnote{\url{https://github.com/meaningfy-ws/rdf-validator-ws/tree/master/resources/templates}}.

\subsubsection{Folder structure}
\begin{lstlisting}
resources/
    templates/          
      html/      <--- custom html report
      json/      <--- custom json report
      query/     <--- queries used by the templates
          validator_query.rq
\end{lstlisting}

\subsection{HTML template variant}
\subsubsection{Folder structure}
\begin{lstlisting}
html/
  config.json   <--- configuration file
  templates/    <--- jinja html templates
    sections/
      simple_view.html
    layout.html
    macros.html
    main.html
\end{lstlisting}


\subsubsection{Template structure}

The HTML template is built be combining four major parts as layout, main, macros and sections. The layout file \textit{layout.html} will have the rules of how the report will look like in terms of positioning and styling. Macros will contain all the jinja2 macros used across the template. A section represents the result of a query that was run with additional html and will be used to build the report. As the name suggest the main file of the html template is main.html. Here is where every other file that are a different section in the report are included and will form the HTML report.

Example of including a section in the main html file:
\begin{lstlisting}

\end{lstlisting}

Each section file has one or more variables where the SPARQL query result is saved as a pandas dataframe.

Example used in \texttt{simple\_view.html}:
\begin{lstlisting}


    
        
        <section class="ui basic segment">
            {{ mc.pandas_table(content, "Validation issues") }}
        </section>
    

\end{lstlisting}

\subsection{JSON template variant}
\subsubsection{Folder structure}

\begin{lstlisting}
json/
  config.json   <--- configuration file
  templates/    <--- jinja json templates
    main.json
\end{lstlisting}

The report will have the following shape:
\begin{lstlisting}
  {
    "focusNode": "http://publications.europa.eu/resource/authority/file-type",
    "message": "Minimum cardinality for <http://www.w3.org/2002/07/owl#versionInfo> is \"1\"^^<http://www.w3.org/2001/XMLSchema#integer>",
    "resultPath": "http://www.w3.org/2002/07/owl#versionInfo",
    "resultSeverity": "http://www.w3.org/ns/shacl#Violation",
    "sourceConstraintComponent": "http://www.w3.org/ns/shacl#MinCountConstraintComponent",
    "sourceShape": "ub2bL232C39",
    "value": null
  }
\end{lstlisting}

\textit{Note:} the brief explanation of fields.
\begin{itemize}
  \item \texttt{focusNode} - Each validation result has exactly one value for the property \texttt{sh:focusNode} that is equal to the focus node that has caused the result. This is the focus node that was validated when the validation result was produced.
  \item \texttt{message} - Validation results may have values for the property \texttt{sh:resultMessage}, for example to communicate additional textual details to humans.
  \item \texttt{resultPath} - Validation results may have a value for the property \texttt{sh:resultPath} pointing at a well-formed SHACL property path.
  \item \texttt{resultSeverity} - Each validation result has exactly one value for the property \texttt{sh:resultSeverity}, and this value is an IRI. The value is equal to the value of \texttt{sh:severity} of the shape in the shapes graph that caused the result, defaulting to sh:Violation if no sh:severity has been specified for the shape.
  \item \texttt{sourceConstraintComponent} - Validation results have exactly one value for the property \texttt{sh:sourceConstraintComponent} and this value is the IRI of the constraint component that caused the result.
  \item \texttt{sourceShape} - Validation results may include, as the only value of the property \texttt{sh:sourceShape}, the shape that the given \texttt{sh:focusNode} was validated against.
  \item \texttt{value} - Validation results may include, as a value of the property \texttt{sh:value}, at most one RDF term that has caused the result.
\end{itemize}

\textit{Note:} make sure that in the templates folder there is a file named the same as the one defined in the \textit{config.json} file (i.e \texttt{"template": "main.json"})

\subsubsection{Template structure}
The JSON report is automatically built by running all queries that are found in the \textit{queries} folder as the system has autodiscover process for this. In the beginning of this report there will be 3 keys that will show the metadata of the report like dataset used, created time and application profile used. Each query result can be identified in the report by the filename and will contain a results key that will represent the result set brought back by the query.
