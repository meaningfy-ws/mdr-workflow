\section{Technical architecture}
\label{sec:technical-architecture}

	\subsection{Prototypical technology structure}
	
	This section presents a generic technology architecture organisation, depicted in Figure \ref{fig:technology-view}, in order to ease interpretation of the diagrams that follow.
	
	\begin{figure}[!h]
		\centering
		\includegraphics[width=.75\textwidth]{images/views/Technology View.png}
		\caption{The prototypical technology view}
		\label{fig:technology-view}
	\end{figure}

	The central concept in the technology view is the node. It represents a computational or physical resource that hosts, manipulates, or interacts with other computational or physical resources. Within a node, various software components are deployed, most foundational of which is the operation system. We say that the node aggregates various software. 
	
	Depending on the chosen level of granularity, the nodes may realise technology services, which represent explicitly defined exposed behaviour. Also, nodes and software may expose interfaces, which represent points of access where a technology service offered by a node can be accessed. The interfaces are constituting parts of the node and we say that an interface may be assigned to a service.
	
	Lastly, the node is connected to the communication network, which represents a set of structures that connects nodes for transmission, routing, and reception of data. There may be one or more networks depending on the organisation security policies and on the internal infrastructure setup. These ArchiMate elements suffice to describe the current and the new technical architectures. 	
	
	\subsection{Current technology architecture}
	\label{sec:technology-current}
	
	The current SU infrastructure necessary to support the application lifecycle is depicted in Figure \ref{fig:technology-current}. It is formed of four nodes a private network and a controlled connection to the internet. 
	
	\begin{figure}[!h]
		\centering
		\includegraphics[width=1.01\textwidth]{images/technology/Current Platform.png}
		\caption{The technology structure that supports the current asset lifecycle}
		\label{fig:technology-current}
	\end{figure}
	
	On the bottom of the diagram is the workstation node which represents the standard machine provided to the members of the SU team by the DIGIT infrastructure unit. The workstation runs Windows 10 \citep{windows10} operating system and has installed on it at least the following set of tools: Excel, Outlook, Chrome (os another modern browser), an SVN client (Tortoise SVN is the default choice in SU team), a specialised XML editor (XML Spy is the default), an FTP client of choice and an SSH client (Putty is the default). This set of software is used to perform all the necessary duties and enact the described business processes. 
	
	On the top of the diagram is depicted the standard service hosting offered by DIGIT. The provided services there are the e-mail service, a dedicated the SVN repository as a service and a set of Jira projects.
	
	In addition, two identical housed Linux servers are provides: one acting as the acceptance environment and the second as the production environment. We call them MDR Linux hosting nodes and they host the legacy workflow management system. This legacy system is based on multiple technologies which were added gradually and organically in the course of its development. It includes components written as Perl scripts, Bash scripts, many XSLT transformation style-sheets, commends relying on existence of an SVN client and Java libraries and scripts.
		
	The MDR Linux hosting nodes serve as testing ground and as operational environment. The nodes run RedHat Linux 7.7 operating system and expose an SSH interface each. This interface is used by the technical team. The technicians connect through it from their workstations and operate the legacy workflow management system remotely.
	
	The communication network is secured as an intranet authenticated connection provided by DIGIT. From intranet a controlled access to the Internet is available through an authenticated proxy. 
	
	This completes the description of the current infrastructure architecture and sets the baseline for the new architecture described in the next section.
	
	\subsection{New technology architecture}
	\label{sec:technology-new}
	
	The new infrastructure architecture is depicted in Figure \ref{fig:technology-new}. What is the same as in the current one is the structure of the workstation node, the intranet and the internet connections and the standard hosted services to which are added VocBench3 and the Adobe FrameMaker server necessary for realising the asset description documentation (ADD) service, and a few other important services within SU among which generation of the inter-institutional style-guide. 
	
	\begin{figure}[!h]
		\centering
		\includegraphics[width=1.01\textwidth]{images/technology/New Platform.png}
		\caption{The technology structure that supports the new asset lifecycle}
		\label{fig:technology-new}
	\end{figure}

	What differs significantly is the disappearance of the hosted nodes (left and right side of Figure \ref{fig:technology-current}) and their replacement with dedicated Amazon Web Service (AWS) hosting, depicted as a large node on the right side of Figure \ref{fig:technology-new}. This node exposes a management interface which provides the possibility of configuring and launching multiple virtual servers (called EC2 instances) of various size and performance capabilities. This puts an administration overhead on the SU team, but also frees it from a large number of constraints experience in the past years this way smoothening the path for this and future digital transformations.
	
	In the new architecture, to support the application architecture described in Section \ref{sec:application-architecture}, two instances are necessary: one optimised for performance and another optimised for persistence. The first one is hosting the operational application components; while the second one triple stores and potentially other types of databases. This initial separation into two could be further optimised by moving some services to dedicated EC2 instances if deemed necessary. Also these two instances are considered acceptance environment where the new application is assembled and tested. Once it is ready to move into production an identical copy of EC2 instances is created and those are further used for the production operations. 
	
	The EC2 instance on the top, marked for performance, runs Ubuntu operating system version 20 (server edition) and has installed the following software : RDF differ (custom built component), RDF validator (custom built component), RDF fingerprinter (custom built component) and Report generator (custom built component), LinkedPipes ETL, SPARQL runner client (custom built component), Jenkins automation system, and Camunda BPM. 
	
	The EC2 instance on the bottom, marked for persistence, runs Ubuntu operating system as well, and hosts initially two triple stores: Fuseki or AnzoGraph which are optimised for operational performance and an instance of Virtuoso triple store meant for long term preservation of RDF data. 
	
	Each of the EC2 instances exposes an SSH interface which can be accessed by the technical team to configure, manipulate and operate the servers, just like in the currently hosted MDR Linux server. 
	