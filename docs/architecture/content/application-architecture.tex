\section{Application architecture}
\label{sec:application-architecture}

	\subsection{Prototypical application structure}
	
	This section presents the application architecture from the solution architecture point of view. A generic solution architecture is depicted in Figure \ref{fig:application-view}.
	
	The application architecture we present covers the application as a ``white box'': its internal component structure, services and interfaces with adjacent applications. Typically the solutions architecture takes the technology aspects into account, accounting for parts of the infrastructure.
	
    \begin{figure}[h]
		\centering
		\includegraphics[width=.5\textwidth]{images/views/Application view.png}
		\caption{The prototypical application structure view}
		\label{fig:application-view}
	\end{figure}

	The central element of the application architecture is the \textit{application service}, which represents an application behaviour or functionality. The application services, from an inter-layer perspective, serve the processes in the business layer and provide support for their realisation. 
	
	
	The application services are realised through application processes. The processes have application components assigned to them signifying their place of encapsulation. Application components are modular and replaceable blocks encapsulating 
	implementation of application services and functionalities. In practice, for clarity, we take a shortcut, and say that the application services are realised through \textit{application components} directly.
	
	Components are said to expose interaction \textit{interfaces} which are modelled, in ArchiMate, as proper parts of the components. The interfaces are assigned to services signifying how the latter are to be accesses and consumed. 
	
	Also components, and processes they encapsulate, access \textit{data objects}, which are the passive components of the application architecture.

	The solution architecture presented in this section is an adaptation of the generic one. Here we focus on presenting for each business process, what application services are used to support it. Moreover, we are interested in grasping the difference in the application layer, between the current and new version of the business processes. 
	
	To do so we split the application view diagrams into three vertical lanes. The left lane hosts the current version of the business process and the application services and components that are used to support it. In the right lane, we place the new business process and the new application services and components that will have to be adopted for the digital transformation. The middle lane, hosts the services and components that are are currently employed and will be carried over into the new application architecture, they are common to the current and the new architectures.

	Next we present an overview of the application architecture, in terms of services alone, depicting how the asset lifecycle stages are served.
	
	\subsection{Current and new application service architecture}
	\label{sec:application-overview}
	
	This section presents how the asset lifecycle stages are supported by the application services. The current application architecture is depicted in Figure \ref{fig:application-current} while the new one in Figure \ref{fig:application-new}. The services employed in each of the architectures are mostly the same, with a few exceptions. What differs more significantly is their utilisation in the asset lifecycle stages. For this reason we set them side by side to provide a contrasting image. 
	
	\begin{figure}
		\centering
		\begin{minipage}{0.485\textwidth}
			\centering
			\includegraphics[width=1.3\textwidth]{images/application/Application Services (current).png}
			\caption{The application services that serve the current asset lifecycle}
			\label{fig:application-current}
		\end{minipage}\hfill
		\begin{minipage}{0.485\textwidth}
			\centering
			\includegraphics[width=1.3\textwidth]{images/application/Application Services (new).png}
			\caption{The application services that serve the new asset lifecycle}
			\label{fig:application-new}
		\end{minipage}
	\end{figure}
	
%	\begin{figure}[h]
%		\centering
%		\includegraphics[width=.75\textwidth]{images/application/Application Services (current).png}
%		\caption{The application services that serve the current asset lifecycle}
%		\label{fig:application-current}
%	\end{figure}

	We start the description from the top of the diagram following the sequence of process stages. First is the \textit{e-mail service}. It represents the entire set of capabilities for sending emails covering both, the email server and the email client. This si realised via the Outlook system \citep{outlook}. 
	
	The \textit{issue management} service represents the capability of recording, documenting, and analysing a change request case in a distributed collaborative manner between multiple roles and actors. This is realised via the Jira system \citep{jira}. 
	
	\textit{Asset content editing} is the service which enables the documentalists in the team to modify asset content and hence implement the request cases. This service is realised by Excel desktop software \citep{excel}.
	
	The \textit{asset content conversion} service implements the conversion between Excel workbook and CAT-XML forms of the asset content. The two representations are equally expressive and the conversion process runs in both directions $ Excel \rightarrow XML $ and $XML \rightarrow Excel $.
	
	\textit{Version diffing} is a service which calculates and presents the difference between two versions of the same asset. The \textit{data validation} service, is self describing. It checks whether the asset content is correct. The asset content conversion, version diffing and data validation services are realised by components that compose the legacy workflow, which is currently in use.
	
	\textit{Asset document management} is provides capabilities for describing, storing and producing documents about assets in various human readable formats (e.g. HTML, PDF, docx). It also allows for document metadata management. Currently this capability is realised by PDF forms, called Asses Document Description (ADD), which are produced from XML-DITA sources \citep{dita-day2005introduction, dita-spec} using the Adobe FrameMaker \citep{framemaker} . 
	
	\textit{Asset metadata \& workflow management} is a service which centralises the flow of the automated processes. It is currently realised by the legacy workflow implementation. 
	
	The \textit{file management \& versioning} service enables storing the content and and tracing the evolution of assets. This service is realised by the SVN system \cite{svn}.
	
	The \textit{data transformation} service is self describing. It stands for a generic capability of transforming data from one form and format into another one. Currently this service is realised by a multitude of custom build transformation procedures, some of which are chained to produce the desired outcome. This service is realised by components that are part of the legacy system. 
	
	The \textit{asset packaging} service prepares the assets for transmission to the partner distribution systems, especially Cellar. It covers functionalities including assembling the necessary asset representations together with their documentation, generating the necessary technical metadata, and zipping the assets in a manner that is acceptable for the partner distribution systems. The type of package and the asset metadata description varies, yet among the most prominent ones are METS \citep{mets}, IMMC\footnote{see \url{https://op.europa.eu/en/web/eu-vocabularies/immc}} and DCAT \cite{dcat2}. 
	
	The last service to mention, that is employed in the current lifecycle process is the \textit{impact assessment \& release note generation} service. Its title is again self explanatory. The value it provides are the release notes, that are published with assets and describe the list of changes implemented in each version. The impact assessment is a type of report, which includes assessments for target stakeholders, which usually are service provides whose systems depend on the assets published by SU. These special impact assessments may target whether a given aspects of an asset have changed which may disrupt functioning systems. So a series of specific checkers and reports are produced to safeguard the partners. 
	
	The new application architecture (Figure \ref{fig:application-new}), as mentioned above, re-configures how services are used across process stages. It replaces the asset content editing service by a new one which is \textit{asset content management}. This service is realised by a fully fledged semantic web editing system -- VocBench3 \citep{stellatovocbench}. In addition two more services are added: asset structure analysis and package transmission services.
	
	The \textit{asset structure analysis} implements an automatic fingerprinting of the asset structure which provides insight into how the managed asset is realised and instantiated. This fingerprinting serves as an indicator for the structural validation in the implementation step. For example, a missing or an extra property will be spotted with this service. 
	
	The \textit{package transmission} provides the possibility to automatically deliver and ingest the prepared packages into the dissemination system, rather than the publication officer performing this operation manually. 
	
%	\subsection{New application service architecture}
%	\label{sec:application-new}	

%	\begin{figure}[h]
%		\centering
%		\includegraphics[width=.75\textwidth]{images/application/Application Services (new).png}
%		\caption{The application services that serve the new asset lifecycle}
%		\label{fig:application-new}
%	\end{figure}	
	
	\subsection{Inception and evolution services and components}
	\label{sec:evolution-application}
	
	In the first stage of the asset lifecycle the technical requirements are limited to client communication and the request documentation services as depicted in Figure \ref{fig:application-inception-evolution}.
	
	\begin{figure}[h]
		\centering
		\includegraphics[width=.6\textwidth]{images/application/InceptionEvolution.png}
		\caption{The application services and components that serve the current and new inception and evolution stage}
		\label{fig:application-inception-evolution}
	\end{figure}
	
	In the left and the right lane of the diagram are places the current and the new lifecycle stages represented as red rectangles. In the middle lane are placed the email and issue management services because they are involved in both, the new and the current architecture. 
	The email service is realised by the Outlook software. The issue management is realised by the Jira system. No changes are foreseen in the way these services are realised in the future. 

	\subsection{Implementation services and components}
	\label{sec:implementation-application}	
	
	Implementation stage is the first place where considerable differences are visible in the way the application architecture is organised. This is depicted in Figure \ref{fig:application-implementation}.
	
	\begin{figure}[!h]
		\centering
		\includegraphics[width=.9\textwidth]{images/application/Implementation v3.png}
		\caption{The application services and components that serve the current and new implementation stage}
		\label{fig:application-implementation}
	\end{figure}
	
	We start discussing this architecture by covering the common services first, situated in the middle lane of the diagram, and then we address particularities, which are situated in the left and right sides of the diagram. 
	
	Issue management system is involved and is realised the same way as in the inception/evolution stage (Section \ref{sec:evolution-application}). In the following sections the services and their realisation that have been already discussed will no longer be addressed.  
	
	The file management \& versioning is realised through SVN \cite{svn}. In the current application architecture, this serves as a foundation for process automation, the triggers being any operations within the SVN repository. The mechanisms employed are similar to what an automation server is doing in the software development content: triggering automatic building, testing, and deploying, facilitating continuous integration and continuous delivery. No such automation server is, however, used due to infrastructure limitations that existed in the past and a custom SVN handler component was developed within the legacy workflow system. 
	
	In the new architecture, the possibility to continue the same practice, of automating process executions based on SVN triggers, is still there. Moreover, we recommend replacing the current SVN handler by an automation system, such as Jenkins \citep{jenkins} and Bamboo \cite{bamboo}. This suggestion is not depicted in the diagram because the current workflow SVN handler, could still serve in the immediate future, but for the long run, we recommend an alternative.
	
	Asset metadata and workflow design management is centralised in a custom built XML file called the ``Vocabulary table``. The vocabulary table is a file which contains descriptions of the currently administered digital assets, descriptions of generic process execution flows and decision rules for special assets. For this reason we say the it is ``workflow design management'' and not the actual ``workflow management'',  because we deal here with description and configuration of the execution paths and decision rules (plus asset metadata), which are enacted by the legacy workflow system. This aggregation of two responsibilities complicates the maintenance and evolution both and shall be separated in the future.  
	
	Later, after adoption of the digital transformation proposed in this architecture, the asset metadata and workflow design management service needs to be broken down in two: the asset metadata management and the workflow management. The metadata management can be realised through a DCAT \citep{dcat2} gnostic cataloguing software. 
	
	The workflow management could be taken over by any workflow management and execution engine implementing service orchestration\footnote{In system administration, orchestration is the automated configuration, coordination, and management of computer systems and software. A number of tools exist for automation of server configuration and management, including Ansible, Puppet, Salt, Terraform, and AWS CloudFormation.} or choreography\footnote{Service choreography is a form of service composition in which the interaction protocol between several partner services is defined from a global perspective.}. 
	
	This could be taken one step further towards a BPMN \citep{bpmn-introduction} execution engine such as Camunda BPM, Flowable and Bonita BPM. There is a installation and configuration overhead for such a system, but the investment is worth considering knowing that they ship with tools for creating workflow and decision models, operating deployed models in production, and allowing users to execute workflow tasks assigned to them. 
	
	Now lets get out attention to the top of the middle lane, to the data validation and version diffing services. As you see the components that realise them are split between the left and right lanes. This means that currently the validation is performed using the XML validator and the version diffing is performed using XML differ; while in the new application, the validation is realised using a SHACL \citep{shacl-spec} validator and version diffing using an RDF differ. The SHACL validator and RDF differ are custom components developed for the new application context.
	
	On the right lane, the asset content editing and the asset content conversion services were introduced in Section \ref{sec:application-overview}. They are the gateway for the asset authoring officers to implement the change request cases. Content editing is realised through Excel software running on the documentalists workstations. Once the edit is finished, the updates are committed to the common SVN repository. The changes a picked up by the SVN handler in the legacy workflow system and trigger a conversion process from Excel to XML and then back to Excel. The conversion is realised by another legacy component implemented to exactly that. The resulted conversion is committed into SVN repository again. This circular conversion, as explained in Section \ref{sec:implementation-current} is necessary because the main source of the assets is represented as CAT-XML, whereas Excel workbook is merely an extension that enables user friendly editing capabilities.
	
	In the new version of the implementation stage, the main source source of the assets is represented as SRC-AP \cite{src-ap-vb3}, a particular a form of SKOS \citep{skos-spec} and is serialised as RDF \citep{rdf11}. The editing is covered by the asset content management service, which is realised by the VocBench3 system \citep{stellatovocbench}. When the authoring officer finishes the implementation of a request case, the content is exported from VocBench3 and committed into the SVN common repository. This commit triggers automatic SHACL validation, RDF diffing and structure analysis operations, which write teh results back into the common repository for the editor to look at and verify the implementation correctness. 
	
	Asset structure analysis service is realised by the RDF fingerprinter component which is necessary for the quality assessment in this and the next stage of the asset lifecycle. And finally asset documentation management service was described in Section \ref{sec:application-overview}. 
	
	The new, RDF-based components rely mostly on the existence of an RDF triple store service, which is a database ensuring persistence, accessibility and data query. This triple store component is depicted not as realising any service in particular but as serving other components: RDF fingerprinter, SHACL validator, RDF differ and VocBench3, so that they function as expected this way creating a dependency between them. We use a generic label ``triple store'' here to express that it is of little important what system is chosen in particular. Among the candidates we can mention Fuseki, Virtuoso, GraphDB, StarDog, AnzoGraph but the list can continue. 
	
	This brings us to the end of the implementation stage description which provided us a parallel between the services and components as they are used currently and how the new application should be implemented. 
	
	\subsection{Pre-release and validation services and components}
	\label{sec:validation-application}
	
	The stages following after the implementation in the current and the new lifecycle processes differ. In Section \ref{sec:lifecycle-new} was described that in the current lifecycle it is the pre-release stage while in the new one it is the validation stage. This difference is also reflected on the application architecture in Figure \ref{fig:application-validation}.
	
	\begin{figure}[h]
		\centering
		\includegraphics[width=.9\textwidth]{images/application/Validation & Pre-release v3.png}
		\caption{The application services and components that serve the current pre-release and the new validation stages}
		\label{fig:application-validation}
	\end{figure}
	
	The current pre-release stage, on the left lane of the diagram, involves asset metadata and workflow design management service, described in Section \ref{sec:implementation-application}, because of the workflow automation involved in this stage. 
	
	The data validation was also described in Section \ref{sec:implementation-application}. One thing that was not mentioned is that the validation relies on an XSLT \citep{xslt3-Kay} controller component which triggers the XML validator component and the XSLT transformation component realising the data transformation service.
	
	The data transformation service 

	\subsection{Release services and components}
	\label{sec:release-application}	
	
	\begin{figure}[h]
		\centering
		\includegraphics[width=.9\textwidth]{images/application/Release v3.png}
		\caption{The application services and components that serve the current and new release stage}
		\label{fig:application-release}
	\end{figure}
	
	\subsection{Publication services and components}
	\label{sec:publication-application}	
	
	\begin{figure}[h]
		\centering
		\includegraphics[width=.9\textwidth]{images/application/Publication v3.png}
		\caption{The application services and components that serve the current and new publication stage}
		\label{fig:application-publication}
	\end{figure}

	