\section{Conclusions}
\label{sec:conclusions}
	
 Presenting the last word on the issues you raised in your paper. Just as the introduction gives a first impression, the conclusion offers a chance to leave a lasting impression. We do this, for example, by highlighting key points in the analysis and findings.
 
 Summarizing your thoughts and conveying the larger implications of your study. The conclusion is an opportunity to succinctly answer the "so what?" question by placing the study within the context of past research about the topic you've investigated.
 
 Demonstrating the importance of your ideas. Don't be shy. The conclusion offers you a chance to elaborate on the significance of your findings.
 
 Introducing possible new or expanded ways of thinking about the research problem. This does not refer to introducing new information [which should be avoided], but to offer new insight and creative approaches for framing/contextualizing the research problem based on the results of your study.
 
 Move towards becoming a data centric organisation
 Adopt a data governance strategy; implement data steward and data custodian roles; training and clear communication in place
 
 Extend the enterprise architecture thinking: fully assess and document motivations and develop a strategy (i.e. include EC vision, EU Directives, data governance initiative, etc.); asses and document in detail who are the main classes of clients and stakeholders; fully assess and document the currently provided services and those needed or requested by clients but not yet implemented; asses and document the technological capabilities and resources necessary to fulfill the service provisions; develop an implementation strategy to move from current state of affairs sto the desired one.
 
 Adopt/implement a monitoring system (technical and business); further extend it with measurements of performance indicators
 
 Further develop missing services and and increase integration and harmonisation of the employed micro services into a holistic reference data management system
 
 Asses the current state of the reference data assets; their structure and content; implement/define clear model for each dataset 
 
 Clean up the datasets to provide internal consistency 
 
 \subsection{Data stewards and custodians}
 Data stewards and custodians
 In Data Governance groups, responsibilities for data management are increasingly divided between the business process owners and information technology (IT) departments. Two functional titles commonly used for these roles are Data Steward and Data Custodian.
 
 Data Stewards are commonly responsible for data content, context, and associated business rules. Data Custodians are responsible for the safe custody, transport, storage of the data and implementation of business rules. Simply put, Data Stewards are responsible for what is stored in a data field, while Data Custodians are responsible for the technical environment and database structure.
 
 The data custodians are the agents that ensure that the content is coherent and complete. It is also responsible for the content correctness and harmonization among multiple stakeholders and its usefulness in broader context of application.
 