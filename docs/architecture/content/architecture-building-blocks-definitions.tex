\section{Architecture building blocks}
\label{sec:building-blocks}

	% TODO: overview and Definitions

	\subsection{Methodology}
	
	In this document we take an enterprise architecture perspective and aim to provide several architecture aspects which are necessary and sufficient to describe the publishing process. 
	
	In developing this architecture, we adopt parts of the TOGAF \citep{togaf9.2}, which is a framework for enterprise architecture that provides an approach for designing, planning, implementing, and governing an enterprise information technology architecture. 
	
	For the architecture representation, we adopt ArchiMate language \citep{archimate3.1}, which is an open and independent enterprise architecture modeling language to support the description, analysis and visualization of architecture within and across business domains in an unambiguous way.
	
	%	TODO: rewrite
	The motivation view is developed by implementing the following steps. First, TOGAF \& ArchiMate were chosen as frameworks. Then some interviews were conducted with the SU management, having as a goal to elicit who are the main stakeholders and their motivations. The interview notes have been distilled and organised in ArchiMate diagrams (using EnterpriseArchitect Tool for drawing the models). Finally this report was written presenting the resulting motivation structure. 
	
	\subsection{ArchiMate elements}
	

	% Please add the following required packages to your document preamble:
% \usepackage{booktabs}
% \usepackage{longtable}
% Note: It may be necessary to compile the document several times to get a multi-page table to line up properly
\begin{longtable}[c]{@{}lll@{}}
	\caption{Overview of the relevant motivation elements \citep{archimate3.1}}
	\label{tab:motivation}\\
	\toprule
	\textbf{Element} & \textbf{Definition} & \textbf{Notation} \\* \midrule
	\endfirsthead
	%
	\multicolumn{3}{c}%
	{{\itshape Table \thetable\ continued from previous page}} \\
	\endhead
	%
	\bottomrule
	\endfoot
	%
	\endlastfoot
	%
			Stakeholder & \parbox{.56\linewidth}{Represents the role of an individual, team, or organization (or classes thereof) that represents their interests in the effects of the architecture.} &     \cincludegraphics[height=2.5\normalbaselineskip]{images/views/elements/stakeholder}   \\

			Driver & \parbox{.56\linewidth}{Represents an external or internal condition that motivates an organization to define its goals and implement the changes necessary to achieve them.} & \cincludegraphics[height=2.5\normalbaselineskip]{images/views/elements/driver}  \\
			
			Assessment & \parbox{.56\linewidth}{Represents the result of an analysis of the state of affairs of the enterprise with respect to some driver.} & \cincludegraphics[height=2.5\normalbaselineskip]{images/views/elements/assesment}   \\
			
			Goal & \parbox{.56\linewidth}{Represents a high-level statement of intent, direction, or desired end state for an organization and its stakeholders.} & \cincludegraphics[height=2.5\normalbaselineskip]{images/views/elements/goal}  \\
			\bottomrule

	\end{longtable}
	
	% Please add the following required packages to your document preamble:
	% \usepackage{booktabs}
	% \usepackage{longtable}
	% Note: It may be necessary to compile the document several times to get a multi-page table to line up properly
	\begin{longtable}[c]{@{}lll@{}}
		\caption{Overview of the relevant business layer elements \citep{archimate3.1}}
		\label{tab:business}\\
		\toprule
		\textbf{Element} & \textbf{Definition} & \textbf{Notation} \\* \midrule
		\endfirsthead
		%
		\multicolumn{3}{c}%
		{{\itshape Table \thetable\ continued from previous page}} \\
		\endhead
		%
		\bottomrule
		\endfoot
		%
		\endlastfoot
		%
			Business actor & \parbox{.5\linewidth}{Represents a business entity that is capable of performing behavior.} & \cincludegraphics[height=1.82\normalbaselineskip]{images/views/business-elements/actor} \\
			Business role & \parbox{.5\linewidth}{Represents the responsibility for performing specific behavior, to which an actor can be assigned, or the part an actor plays in a particular action or event.} & \cincludegraphics[height=1.82\normalbaselineskip]{images/views/business-elements/role} \\
			\parbox{.1\linewidth}{Business collaboration} & \parbox{.5\linewidth}{Represents an aggregate of two or more business internal active structure elements that work together to perform collective behavior.} & \cincludegraphics[height=1.82\normalbaselineskip]{images/views/business-elements/collaboration} \\
			Business interface & \parbox{.5\linewidth}{Represents a point of access where a business service is made available to the environment.} & \cincludegraphics[height=1.82\normalbaselineskip]{images/views/business-elements/interface} \\
			Business process & \parbox{.5\linewidth}{Represents a sequence of business behaviors that achieves a specific result such as a defined set of products or business services.} & \cincludegraphics[height=1.82\normalbaselineskip]{images/views/business-elements/process} \\
			Business function & \parbox{.5\linewidth}{Represents a collection of business behavior based on a chosen set of criteria (typically required business resources and/or competencies), closely aligned to an organization, but not necessarily explicitly governed by the organization.} & \cincludegraphics[height=1.82\normalbaselineskip]{images/views/business-elements/function} \\
			Business event & \parbox{.5\linewidth}{Represents an organizational state change.} & \cincludegraphics[height=1.82\normalbaselineskip]{images/views/business-elements/event} \\
			Business service & \parbox{.5\linewidth}{Represents explicitly defined behavior that a business role, business actor, or business collaboration exposes to its environment.} & \cincludegraphics[height=1.82\normalbaselineskip]{images/views/business-elements/service} \\
			Business object & \parbox{.5\linewidth}{Represents a concept used within a particular business domain.} & \cincludegraphics[height=1.82\normalbaselineskip]{images/views/business-elements/object} \\
			Representation & \parbox{.5\linewidth}{Represents a perceptible form of the information carried by a business object.} & \cincludegraphics[height=1.82\normalbaselineskip]{images/views/business-elements/representation} \\ \bottomrule
		
	\end{longtable}

	% Please add the following required packages to your document preamble:
	% \usepackage{booktabs}
	% \usepackage{longtable}
	% Note: It may be necessary to compile the document several times to get a multi-page table to line up properly
	\begin{longtable}[c]{@{}lll@{}}
		\caption{Overview of the relevant application layer elements \citep{archimate3.1}}
		\label{tab:application}\\
		\toprule
		\textbf{Element} & \textbf{Definition} & \textbf{Notation} \\* \midrule
		\endfirsthead
		%
		\multicolumn{3}{c}%
		{{\itshape Table \thetable\ continued from previous page}} \\
		\endhead
		%
		\bottomrule
		\endfoot
		%
		\endlastfoot
		%
			\parbox{.1\linewidth}{Application component} & \parbox{.5\linewidth}{Represents an encapsulation of application functionality aligned to implementation structure, which is modular and replaceable.} & \cincludegraphics[height=1.82\normalbaselineskip]{images/views/application-elements/component} \\
			\parbox{.1\linewidth}{Application interface} & \parbox{.5\linewidth}{Represents a point of access where application services are made available to a user, another application component, or a node.} & \cincludegraphics[height=1.82\normalbaselineskip]{images/views/application-elements/interface} \\
			\parbox{.1\linewidth}{Application function} & \parbox{.5\linewidth}{Represents automated behavior that can be performed by an application component.} & \cincludegraphics[height=1.82\normalbaselineskip]{images/views/application-elements/function} \\
			\parbox{.1\linewidth}{Application process} & \parbox{.5\linewidth}{Represents a sequence of application behaviors that achieves a specific result.} & \cincludegraphics[height=1.82\normalbaselineskip]{images/views/application-elements/process} \\
			\parbox{.1\linewidth}{Application event} & \parbox{.5\linewidth}{Represents an application state change.} & \cincludegraphics[height=1.82\normalbaselineskip]{images/views/application-elements/event} \\
			\parbox{.1\linewidth}{Application service} & \parbox{.5\linewidth}{Represents an explicitly defined exposed application behavior.} & \cincludegraphics[height=1.82\normalbaselineskip]{images/views/application-elements/service} \\
			\parbox{.15\linewidth}{Data object} & \parbox{.5\linewidth}{Represents data structured for automated processing.} & \cincludegraphics[height=1.82\normalbaselineskip]{images/views/application-elements/object} \\ \bottomrule		
		
	\end{longtable}

	% Please add the following required packages to your document preamble:
	% \usepackage{booktabs}
	% \usepackage{longtable}
	% Note: It may be necessary to compile the document several times to get a multi-page table to line up properly
	\begin{longtable}[c]{@{}lll@{}}
		\caption{Overview of the relevant technology layer elements \citep{archimate3.1}}
		\label{tab:technology}\\
		\toprule
		\textbf{Element} & \textbf{Definition} & \textbf{Notation} \\* \midrule
		\endfirsthead
		%
		\multicolumn{3}{c}%
		{{\itshape Table \thetable\ continued from previous page}} \\
		\endhead
		%
		\bottomrule
		\endfoot
		%
		\endlastfoot
		%
			Node & \parbox{.5\linewidth}{Represents a computational or physical resource that hosts, manipulates, or interacts with other computational or physical resources.} & \cincludegraphics[height=1.82\normalbaselineskip]{images/views/technology-elements/node} \\
			Device & \parbox{.5\linewidth}{Represents a physical IT resource upon which system software and artifacts may be stored or deployed for execution.} & \cincludegraphics[height=1.82\normalbaselineskip]{images/views/technology-elements/device} \\
			\parbox{.15\linewidth}{System software} & \parbox{.5\linewidth}{Represents software that provides or contributes to an environment for storing, executing, and using software or data deployed within it.} & \cincludegraphics[height=1.82\normalbaselineskip]{images/views/technology-elements/software} \\
			\parbox{.15\linewidth}{Technology interface} & \parbox{.5\linewidth}{Represents a point of access where technology services offered by a node can be accessed.} & \cincludegraphics[height=1.82\normalbaselineskip]{images/views/technology-elements/interface} \\
			\parbox{.18\linewidth}{Communication network} & \parbox{.5\linewidth}{Represents a set of structures that connects nodes for transmission, routing, and reception of data.} & \cincludegraphics[height=1.82\normalbaselineskip]{images/views/technology-elements/network} \\
			\parbox{.15\linewidth}{Technology service} & \parbox{.5\linewidth}{Represents an explicitly defined exposed technology behavior.} & \cincludegraphics[height=1.82\normalbaselineskip]{images/views/technology-elements/service} \\
			Artifact & \parbox{.5\linewidth}{Represents a piece of data that is used or produced in a software development process, or by deployment and operation of an IT system.} & \cincludegraphics[height=1.82\normalbaselineskip]{images/views/technology-elements/artifact} \\ \bottomrule
		
	\end{longtable}



	\begin{longtable}[c]{@{}lll@{}}
	\caption{Overview of the ArchiMate relationships \cite{archimate3.1}}
	\label{tab:relations}\\
	\toprule	
	\textbf{Element} & \textbf{Definition} & \textbf{Notation} \\* \midrule
	\endfirsthead
	%
	\multicolumn{3}{c}%
	{{\itshape Table \thetable\ continued from previous page}} \\
	\endhead
	%
	\bottomrule
	\endfoot
	%
	\endlastfoot
	%
			\multicolumn{2}{c}{\textbf{Structural Relationships}} & \multicolumn{1}{l}{\textbf{}} \\ 
			Composition & \parbox{.56\linewidth}{Represents that an element consists of one or more other concepts.} & \cincludegraphics[width=4\normalbaselineskip]{images/views/relations/composition} \\
			
			Aggregation & \parbox{.56\linewidth}{Represents that an element combines one or more other concepts.} & \cincludegraphics[width=4\normalbaselineskip]{images/views/relations/aggregation} \\
			Assignment & \parbox{.56\linewidth}{Represents the allocation of responsibility, performance of behavior, storage, or execution.} & \cincludegraphics[width=3.5\normalbaselineskip]{images/views/relations/assignment} \\
			Realization & \parbox{.56\linewidth}{Represents that an entity plays a critical role in the creation, achievement, sustenance, or operation of a more abstract entity.} & \cincludegraphics[width=3.5\normalbaselineskip]{images/views/relations/realisation} \\
			\multicolumn{2}{c}{\textbf{Dependency Relationships}} & \multicolumn{1}{l}{\textbf{}} \\ 
			Serving & \parbox{.56\linewidth}{Represents that an element provides its functionality to another element.} & \cincludegraphics[width=4\normalbaselineskip]{images/views/relations/serving} \\
			Access & \parbox{.56\linewidth}{Represents the ability of behavior and active structure elements to observe or act upon passive structure elements.} & \cincludegraphics[width=4\normalbaselineskip]{images/views/relations/access} \\
			Influence & \parbox{.56\linewidth}{Represents that an element affects the implementation or achievement of some motivation element.} & \cincludegraphics[width=4\normalbaselineskip]{images/views/relations/influence} \\
			Association & \parbox{.56\linewidth}{Represents an unspecified relationship, or one that is not represented by another ArchiMate relationship.} & \cincludegraphics[width=4\normalbaselineskip]{images/views/relations/association} \\
			\multicolumn{2}{c}{\textbf{Dynamic Relationships}} & \multicolumn{1}{l}{\textbf{}} \\ \midrule
			Triggering & \parbox{.56\linewidth}{Represents a temporal or causal relationship between elements.} & \cincludegraphics[width=4\normalbaselineskip]{images/views/relations/triggers} \\
			Flow & \parbox{.56\linewidth}{Represents transfer from one element to another.} &  \cincludegraphics[width=4\normalbaselineskip]{images/views/relations/flows}\\
			\multicolumn{2}{c}{\textbf{Other Relationships}} & \multicolumn{1}{l}{\textbf{}} \\ \midrule
			Specialization & \parbox{.56\linewidth}{Represents that an element is a particular kind of another element.} & \cincludegraphics[width=3.5\normalbaselineskip]{images/views/relations/specialises} \\			
			Junction & \parbox{.56\linewidth}{Used to connect relationships of the same type.} & \cincludegraphics[width=4\normalbaselineskip]{images/views/relations/junction} \\ \bottomrule
	\end{longtable}

	\subsection{Architecture views}
	\label{sec:views}
	