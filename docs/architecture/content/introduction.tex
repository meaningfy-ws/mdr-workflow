\section{Introduction}
\label{sec:introduction}
%	TODO: About Standardisation Unit (SU): where it is, what it currently does, and why. 

	
	This document provides a working definition of the architectural stance and design decisions that are to be adopted for the asset publication life-cycle process. This process is materialised as the publication workflow and is currently employed by the Standardisation Unit (SU) at the Publications Office of the European Union (PO).
	
	In this document we (a) establish the baseline architecture, supported by  strategic and motivational information; and (b) develop a target architecture guiding the digital transformation processes towards new technologies. This constitutes a natural evolution in response to changing mission needs defined by SU management, and also takes into consideration the strategic directions proposed by the PO, the European Commission (EC) and the European Parliament (EP) as presented below.
	
	\subsection{Background considerations}
	
	Given the increasing importance of data standards for the EU institutions, a number of initiatives driven by the public sector, industry and academia have been kick-started in recent years. Some have grown organically, while others are the result of standardisation work. The vocabularies and semantics that they are introducing, together with the technologies that they are using, all differ. These differences hamper data interoperability and thus its reuse by them or by the wider public. This creates the need for a common data standard for publishing public reference data and models, hence allowing data from different sources to be easily accessed, linked, and consequently reused.
	
	The PSI directive \cite{directive-2019/1024} across the EU calls for open, unobstructed access to public data in order to improve transparency and to boost innovation via the reuse of public data. The reference data maintained and published by the PO has been identified as data with a high-reuse potential \cite{d-high-value-assets}. Therefore, making this data available in machine-readable formats, as well as following the data as a service paradigm, are required in order to maximise its reuse.
	
	In this context, the Publications Office of the European Union maintains and publishes an ever-increasing number of reference data assets which are vital in the context of inter-institutional information exchange. With regards to reference data, the PO provides an ever-increasing number of services to the main institutional stakeholders and with the aim to extend them to a broader public, enabling active or passive participation in the reference data life cycle, standardisation and harmonisation.

	\subsection{EU trajectory towards public sector linked open data}
	
	European institutions started out to adopt Semantic Web and Linked Data technologies as part of their visions to become data-centred e-government bodies \citep{decission-456/2005/EC,decission-2015/2240}. 
	
	The EU institutions also aim for implementation of a single digital gateway to ``facilitate interactions between citizens and businesses, on the one hand, and competent authorities, on the other hand, by providing access to online solutions, facilitating the day-to-day activities of citizens and businesses and minimising the obstacles encountered in the internal market. The existence of a single digital gateway providing online access to accurate and up-to-date information, to procedures and to assistance and problem-solving services could help raise the users' awareness of the different existing online services and could save them time and expense'' \citep{directive-2018/1724}. This is well in line with earlier established goals for encouraging the open data and the re-use of public sector information \citep{directive-2013/37/EU,directive-2019/1024}.

	Many of the legacy systems used in the EU institutions use XML data format for exchange and document formats governed by the XSD schemes \citep{xsd1.1-spec}. The aim is to evolve technologically so that both existing and new systems are capable to operate with semantic data representations using RDF \citep{rdf11}, OWL \citep{owl2.0,owl2}, SHACL \citep{shacl-spec} and other representations, and serialised at least in RDF/XML \citep{rdf-xml-Beckett:04:RSS,rdf-xml-Schreiber:14:RXS}, Turtle \citep{turtle-Carothers:14:RT} and JSON-LD \citep{spornyjson,sporny2014json} formats.
	
	For this reason, the PO has already been publishing data in RDF format for over a decade using the Cellar repository \citep{cdm-francesconi2015ontology}. Also, the SU, in particular, is committed to publish and disseminate reference data in semantic formats. 
	
	%Next we outline the state of affairs of the SU to describe the context of the current work. 
	
	\subsection{Target audience}
	\label{sec:audience}
	The target audience for this document comprises the following groups and stakeholders:	

	\begin{itemize}
		\item Management of the SU
		\item Enterprise architects and data governance specialists
		\item Documentalists involved in the reference data life-cycle
		\item Technical staff in charge of operating workflow components
		\item Developers in charge of workflow and component implementation
		\item Third parties using the SU services and data in charge of harmonisation and standardisation of metadata and processes
	\end{itemize}	
	
	\subsection{Document scope}
	\label{sec:scope}
	
	This document aims to support SU in the transition towards semantic technologies with a particular focus on the architecture of the publication workflow. 
	
	The central use case is to support the asset publication life-cycle detailed in Section \ref{sec:lifecycle-current-stages}. This includes managing the incoming requests, editing the reference assets in VocBench3 system \citep{stellato2017towards,stellatovocbench}, then exporting the RDF data and passing them as input to a set of processes that validate, assess, transform, package and finally publish the assets in Cellar \cite{cdm-francesconi2015ontology}, the main dissemination platform. This use case has been broken down into sub-use cases that are detailed in Section \ref{sec:business-use cases}.
	
	This document will provide a motivational, business and application account of the asset life-cycle workflow. Each of these accounts is limited strictly to the success scenario of the above-mentioned use case and does not include possible extensions and variations.
	
	There is a series of aspects that were intentionally left out. For example the recommendations related to the data governance both internally within the SU and also externally in relation with partners, stakeholders and clients. No implementation details are specified for the new components. This shall be covered elsewhere. Little or not account is provided about the data structures and static objects used in the business process or exchanged between the application services. No monitoring or performance measurement system is foreseen by this architecture across all levels starting from motivation level key performance indicators (KPI), through business level process monitoring, down to performance measurement of the applications and the infrastructure indicators. 
	
	This architecture is also focused on the transition from XML-based asset sources representation to the RDF-based representation along with the resulting implications. And does not address in great detail, for instance, how the new impact assessment module shall be implemented; or how the workflow orchestration engine shall be configured, what process automation service to use, or what technologies could be chosen for that. Such decisions shall be carried out in subsequent steps.	No asset inventory and metadata management system is proposed either, it is only identified as missing and necessary in Section \ref{sec:implementation-application}. 
	
	With this scope in mind, we present in the next section what is the general direction that the digital transformation shall take and what is the starting point for it, given the business and technical state of play. 