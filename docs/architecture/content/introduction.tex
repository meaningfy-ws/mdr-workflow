\section{Introduction}
\label{sec:introduction}
%	TODO: About Standardisation Unit (SU): where it is, what it currently does, and why. 
	
	This document provides a working definition of what is the architectural stance and the design decisions that shall be adopted for the reference asset management lifecycle materialised as the publishing workflow used by the Standardisation Unit (SU) at the Publications Office of the European Union (PO).
	
	In this document we (a) establish the baseline architecture, supported by the strategic and motivational information; (b) develop a target architecture guiding the transitional processes of implementing new technologies. This constitutes a natural evolution in response to changing mission needs defined by the SU management, and also takes into consideration the strategic directions proposed by PO and European Commission (EC) and European Parliament (EP).
	
	
	\subsection{Background considerations}
	
	Given the increasing importance of data standards for the European institutions, a number of initiatives driven by the public sector, the industry and academia have been kick started in the recent years. Some have grown organically, while others are the result of standardisation work. The vocabularies and the semantics that they are introducing and the technologies that they are using all differ. These differences hamper data interoperability and thus its reuse by them or by the wider public. This creates the need for a common data standard for publishing public reference data and models, hence allowing data from different sources to be easily accessed and linked, and consequently reused.
	
	PSI directive \cite{directive-2013/37/EU} across the EU is calling for open, unobstructed access to public data in order to improve transparency and to boost innovation via the reuse of public data. The reference data maintained and published by the PO have been identified as data with a high-reuse potential \cite{d-high-value-assets}. Therefore, making this data available in machine-readable formats, following the data as a service paradigm, is required in order to maximise its reuse.
	
	In this context the Publications Office of the European Union maintains and publishes an ever growing number of reference data assets vital in the context of inter-institutional information exchange. With regards to reference data, PO provides an ever growing number of services to the main institutional stakeholders and with the aim to extend them to a broader public; enabling active or passive participation in the reference data life cycle, standardisation and harmonisation.

	\subsection{EU trajectory towards semantic and linked data}
	
	European institutions set sail to adopt Semantic Web and Linked Data technologies as part of the vision to become data centred e-government bodies [cite find directives]. 
	
	Many of the legacy systems in the institutions use XML data format for exchange and document formats governed by the XSD schemes \citep{xsd1.1-spec}. The aim is to evolve so that the existing and the new systems are capable to operate with semantic data representations using RDF \citep{rdf11}, OWL \citep{owl2.0,owl2}, SHACL \citep{shacl-spec} and other representations and serialised at least in RDF/XML \citep{rdf-xml-Beckett:04:RSS,rdf-xml-Schreiber:14:RXS}, Turtle \citep{turtle-Carothers:14:RT} and JSON-LD \citep{spornyjson,sporny2014json} formats.
	
	For this reason, the PO is already publishing the data in RDF format for over a decade using Cellar repository [cite]. And the SU, in particular, is committed to publish and disseminate the reference data in semantic formats. Next we describe the state of affairs of the SU to describe the context of the current work. 
	
	\subsection{Target audience}
	\label{sec:audience}
	The target audience for this document comprises the following groups of stakeholders:	

	\begin{itemize}
		\item Management of the SU
		\item Enterprise architects and data governance specialists
		\item Documentalists involved in the reference data lifecycle
		\item Technical staff in charge of operating the workflow components
		\item Developers in charge of the workflow implementation		
	\end{itemize}	
	
	\subsection{Document scope}
	\label{sec:scope}
	
	This document aims to support SU in the transition towards the semantic technologies focusing on the architecture of the publishing workflow. 
	The central use case is to support the asset management lifecycle presented in Section (below). It includes managing the incoming requests, editing the reference assets in VocBench3 system \citep{stellato2017towards,stellatovocbench}, then exporting the RDF data and passing them as input to a set of processes that validate, assess, transform, package and finally publish the assets in Cellar \cite{cdm-francesconi2015ontology}, the main dissemination platform.
	
	This document will provide a motivational, business and application account of the workflow. Each of these accounts is limited strictly to the success scenario of the above mentioned use case and does not include possible extensions and variations which may be. 
	
%	TODO: write 
%	In the past years much effort was invested into the eProcurement ontology initiative, including definition of requirements, provision of general specifications, identification of the main use cases, and laborious development of a preliminary shared conceptual model expressed using Unified Modelling Language (UML) \cite{uml-userguide,uml2.5}. 
%	
%	%	The next step is to establish a process for generating a formal OWL ontology, the final artefact of the project.  
%	
%	The general methodology for developing the eProcurement ontology is described in \cite[3--15]{d2.01-2017}. It describes a process comprising the following steps:
%	\begin{enumerate}
%		\item\label{step:1} Define use cases
%		\item\label{step:2} Define the requirements for the use cases
%		\item\label{step:3} Develop a conceptual data model
%		\item\label{step:4} Consider reusing existing ontologies
%		\item\label{step:5} Define and implement an OWL ontology		
%	\end{enumerate}
%	
%	The ultimate objective of the eProcurement ontology project is to put forth a commonly agreed OWL ontology that will conceptualise, formally encode and make available in an open, structured and machine-readable format data about public procurement, covering end-to-end procurement, i.e. from notification, through tendering to awarding, ordering, invoicing and payment \citep{d4.07-2016}.
%	
%	Work so far has concentrated on the conceptual modelling of the eNotification phase, taking into consideration the needs of other phases. The UML conceptual model has been created with the forthcoming procurement standard forms (eForms) in mind; the model has not been mapped to the current standard forms.
%	
%	In the 2020 ISA$^2$ work programme a new project has been set up to analyse existing procurement data through the lens of the newly developed conceptual model. This means that the conceptual model needs to be transposed into a formal ontology and a subset of the existing eProcurement data must be transformed into RDF format such that they instantiate the eProcurement ontology and are conform to a set of predefined data shapes. Initially the notification phase is considered, while the subsequent datasets will be decided at a later stage.
%	
%	Working under the assumption that Steps \ref{step:1}--\ref{step:4} have been completed, the current efforts channel on designing, implementing and executing the necessary tasks in order to accomplish Step \ref{step:5} from the above process. 
%	
%	Once the formal ontology is created and the XML data is transformed into RDF representation, the data can be queried in order to validate the suitability to satisfy the business use cases defined in \cite[Sec. 3]{d2.01-2017}.
%	
%	This document comprises of architectural specification and implementation guidelines that shall be taken into consideration when developing the formal ontology. Other related artefacts (i.e. documents, scripts and datasets) are presented in Section \ref{sec:process-approach}, where it is described, in detail, the process for accomplishing the generation the formal eProcurement ontology, transformation of XML data and the ontology validation.
%	
%	There is a number of aspects that are excluded from the scope of this project stage:	
%	\begin{itemize}
%		\item Change management and maintenance of the ontology content.
%		\item Content authoring and conceptual design of the domain model.
%		\item Practical implementation of systems that implement the ontology.
%	\end{itemize}
%	
%	Currently in scope are the following items:
%	\begin{itemize}
%		\item designing an ontology architecture (this document),
%		\item create guidelines and conventions for the UML conceptual model \citep{costetchi2020b}, 
%		\item develop a set of transformation scripts from the UML model into a formal ontology
%		\item implement a set of scripts to transform the existing XML eProcurement data into RDF format,
%		\item put forward a method to validate the generated formal ontology using the current eProcurement data.
%	\end{itemize}