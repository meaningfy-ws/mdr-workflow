%\section{Context of the digital transformation}
\section{Publication workflow digital transformation}
\label{sec:context}


	\subsection{State of play}
	The SU publishes reference data in several formats, the most important being XML \citep{xml1-spec}, XSD \citep{xsd1.1-spec} and RDF/XML \citep{rdf-xml-Beckett:04:RSS,rdf-xml-Schreiber:14:RXS}. On the application side (see Section \ref{sec:application-architecture}), the SU currently employs a legacy (custom-built) system for controlling and executing, in part, the asset management life-cycle operations (legacy workflow system). The system was developed using a mixture of XSLT technology \cite{xslt3-Kay}, Perl and Bash scripting languages. The system was developed to execute a wide variety of conversions and transformations based on XML source files into various other formats including human readable documents.
	 
	The source data representation (XML in this case) has the primary role to serve as the original reference (and the only source of truth) and, additionally, maintaining non-redundancy and rich expressivity\footnote{In computer science, the expressive power (also called expressiveness or expressivity) of a language is the breadth of ideas that can be represented and communicated in that language. The more expressive a language is, the greater the variety and quantity of ideas it can be used to represent.}. All other data forms and representations are secondary and are generated by transformation and conversion processes from the source representation.
	 
	One peculiarity of the legacy system setup is that the editing of the asset content is performed using MS Excel \citep{excel}. This is done by transformation of the content from XML representation into Excel style-sheets, which are edited by SU documentalists and then converted back into XML format. In this way, a circular transformation is achieved which also serves as an integrity checking and validation mechanism. In addition, XSD \citep{xsd1.1-spec} schema definitions are used to validate the XML source representation.
	 
	The legacy workflow system uses the file system for data persistence. In addition, this functionality is aided by a version controlling system, SVN \cite{svn}, to trace the evolution of data across time.
	 
	Some steps in the legacy workflow have been automated. The automation is based on cron jobs\footnote{The software utility cron, also known as cron job, is a time-based job scheduler in a Unix-like computer operating systems} and SVN hooks that, upon changes in the source XML or Ms Excel files, trigger a set of conversion mechanisms.  Some other steps require manual triggering and eventually parameterisation intervention. The execution of the automated steps often requires intervention by technical staff or someone with above-average IT skills which represent an impediment for the non-technical documentalists and a hindrance for the IT staff.
	 
	Moreover, the maintenance of this system is burdened by a technical debt that has accumulated over time, because the system has evolved organically based on incorporation of many requests. 
	

	\subsection{Towards semantic technology workflow}
	
	The SU's mission regarding the technological evolution is to migrate towards Semantic Web and Linked Data technologies and representations. The maintenance of reference data is currently done based on XML source representation and the desired transition is towards RDF-based representation\citep{rdf11,rdf11-semantics}. For that purpose, MS Excel and XML sources are no longer suitable and a dedicated editor is necessary.
	
	To solve this issue, SU took the development flagship of the VocBench3 \citep{stellatovocbench, stellato2017towards} system, a web-based, multilingual, vocabulary editing tool based on the SKOS \citep{skos-spec} model, which is modelled with RDFS \citep{rdfs1-spec,rdfs11-spec}. Later, VocBench3 was developed to support authoring of RDFS \citep{rdfs11-spec} and OWL \citep{owl2.0} vocabularies.
	
	Switching to RDF-based sources and adoption of VocBench3 system implies a technological and business process disruption. The main reason for this is that the legacy workflow system operates only with XML-based sources and does not support RDF sources. RDF representation being only a by-product derived from XML. 
	
	VocBench3 naturally adopted a persistence based on triple stores, which are NoSQL\footnote{A NoSQL database provides a mechanism for storage and retrieval of data that is modelled in means other than the tabular relations used in relational databases.} database systems implementing the directed graph data model instead of the hierarchical or relational data model. The relational data model is mentioned here because the MS Excel worksheets are based on tabular data organisation; the hierarchical data model is mentioned because XML is fundamentally a hierarchically organised data structure; also, both are only partially compatible with the graph paradigm present in semantic data models \citep{rdf-semantics}. 
	
	Migration towards a new workflow that integrates VocBench3 requires reconciliation between file-system and database approaches to persistence. Also, a paradigmatic transition to graph-based data representation in RDF, from the hierarchical models of source representation in XML, and tabular models used for source authoring in Excel, is necessary.
	 
	The legacy workflow system is also lacking in semantic validation, structural analysis (fingerprinting) and content comparison capabilities (calculating the difference between two versions of an asset) for RDF data representation. A transition would imply development of at least these new capabilities in order to maintain business processes similar to the current ones (see Section \ref{sec:business-architecture}). 
	