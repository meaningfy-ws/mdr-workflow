\section{Business use cases}
\label{sec:business-use cases}
	
	This section presents the core business use cases that have been identified in discussions with the SU team members. These use cases have been structured in the light of the new asset lifecycle process, and not the current one, even though they are heavily inspired by the current asset lifecycle process. Section \ref{sec:business-architecture} will present the designs of the current and the new asset lifecycle processes illustrating differences and commonalities. The designed processed are a derived from the use case descriptions following below. 
	
	\subsection{UC1: Evolution: Register request for content change}
	\label{sec:uc1}
	
	\textbf{Trigger:} A request for content change is received in the functional mailbox
	
	\textbf{Success guarantee:} A complete and clear change request case is created and scheduled for approval.
	
	\textbf{Main success scenario:}
	
	\begin{enumerate}
		\item The client requests a change of one or multiple concepts in an asset
		\item The request manager creates a new change request case and acknowledges the client
		\item The asset manager analyses the request (in terms of business needs and in term of data management implications) and summarises the case
		\item Request manager informs the client of the case summary
		\item The request manager proposes the case for discussion in the next meeting of the team steering committee		
	\end{enumerate}
	\textbf{Extensions:}
	\begin{enumerate}
		\item [4a] The request is incomplete or unclear:
		\begin{enumerate}
			\item [4a1] The asset manager formulates the information needs			
			\item [4a2] The request manager collects the needed details and clarifications from the client
			\item [4a3] Return to step 3 
		\end{enumerate}
		\item [4b] The request is complex and needs deeper conceptual analysis and modelling/design:
		\begin{enumerate}
			\item [4b1] The asset manager presents the case to the knowledge modelling expert 
			\item [4b2] The knowledge modelling expert analyses the case and proposes a solution
			\item [4b3] Return to step 3			
		\end{enumerate}
	\end{enumerate}
	
	
	\subsection{UC2: Evolution: Register request for content change}
	\label{sec:uc2}
	
	\textbf{Trigger:} A team steering committee meeting takes place 
	
	\textbf{Preconditions:} A case is in the meeting agenda 
	
	\textbf{Success guarantee:} The case is rejected or approved for implementation
	
	\textbf{Main success scenario:}
	
	\begin{enumerate}
		\item Any time between the case is proposed for discussion and the meeting, committee members may assess the open cases and add business, technical or implementation related comments. 
		\item During the meeting, the asset manager presents the case.
		\item The steering committee members discuss, comment the case
		\item The steering committee approves the case for implementation
		\item Asset manager schedules the case for implementation		
	\end{enumerate}
	\textbf{Extensions:}
	\begin{enumerate}
		\item [4a] The case is rejected:
		\begin{enumerate}
			\item [4a1] The steering committee reject the case along with a justification
			\item [4a2] The request manager informs the client and provides recommendations			
		\end{enumerate}
		\item [4b] The case needs additional input:
		\begin{enumerate}
			\item [4b1]  The steering committee reject the case along with a request for action, information or agreement to an alternative proposal
			\item [4b2] The request manager informs the user and requests additional actions, information or agreement to an alternative proposal
			\item [4b3] The request manager register request for content change (UC1)
		\end{enumerate}
	\end{enumerate}

	\subsection{UC3: Implementation: Implement request for content change}
	\label{sec:uc3}
	
	\textbf{Trigger:} A case is scheduled for implementation
	
	\textbf{Preconditions:} The case is approved for implementation
	
	\textbf{Success guarantee:} The case is implemented and validated, while the data are exported and stored in the common repository
	
	\textbf{Main success scenario:}
	
	\begin{enumerate}
		\item The request manager schedules a case for implementation
		\item Data authoring officer reads the case and executes the content authoring accordingly
		\item Data authoring officer automatically or assisted by the data processing officer
		\begin{enumerate}
			\item exports the asset from the authoring tool and 
			\item runs the SHACL validation for conceptual and structural issues and
			\item runs the difference calculation between exported content and the previous release export and
			\item runs the fingerprint calculation for the exported content
			\item the data and reports are stored in the common repository 
		\end{enumerate}		
		\item Data authoring officer checks 
		\begin{enumerate}
			\item (verification) that the diff report calculated between the previous release export corresponds to the implemented change request\footnote{This is to validate that the export reflects change request case for the change request ticket, keeping the editors on the safe side. If all is good, then this is the final diff.}.
			\item (validation) that no structural anomalies are present in the fingerprint and validation reports
		\end{enumerate}		 
		\item Repeat steps 1 - 4 until all cases are implemented for the asset
		\item Data authoring officer informs the quality assurance officer of the successful implementation of all cases.		
	\end{enumerate}
	
	\textbf{Extensions:}
	\begin{enumerate}
		\item [2a] Translations are necessary:
		\begin{enumerate}
			\item [2a1] Additionally, data authoring officer manages the necessary translations and proof-reading (process described elsewhere: export selected data for translators, send to the translation unit,  import updated data containing the translations, validate and proofread the translations)			
		\end{enumerate}
		\item [4a] Implementation or data is invalid:
		\begin{enumerate}
			\item [4a1] Data authoring officer collects and documents all the issues 
			\item [4a2] Return to step 2			
		\end{enumerate}
	\end{enumerate}
	
	\subsection{UC4: Validation: Validate the request for change}
	\label{sec:uc4}
	
	\textbf{Trigger:} An implementation is scheduled for validation (data available in SRC-AP format along the assessment reports)
	
	\textbf{Preconditions:} All the cases are implemented for an asset and no further updates are foreseen
	
	\textbf{Success guarantee:} The data are validated by a second pair of eyes and marked as fit for publication
	
	\textbf{Main success scenario:} 
	
	\begin{enumerate}
		\item Data authoring officer provides the data and validation reports in the common repository 
		\item Quality assurance office verifies that the diff report corresponds to case requirements
		\item Quality assurance office checks the fingerprint and validation reports for semantic or structural anomalies. 
		\item Quality assurance office accepts the implementation and the data and informs the asset manager
		\item Asset manager marks the asset as fit for publication.
		
	\end{enumerate}
	
	\textbf{Extensions:}
	
	\begin{enumerate}
		\item [4a] Data quality issues are detected:
		\begin{enumerate}
			\item [4a1] The quality assurance officer identifies  and documents issues in the validation and fingerprint reports and informs the data authoring officer what the issues are and eventually explains how to fix them.
			\item [4a2] Data authoring officer implements the request for change (UC3)			
		\end{enumerate}
		\item [4b] Implementation issues are detected:
		\begin{enumerate}
			\item [4b1] The quality assurance officer identifies and documents issues in the diff report and informs the data authoring officer what the issues are and eventually explains how to fix them.
			\item [4b2] Data authoring officer implements the request for change (UC3)			
		\end{enumerate}
	\end{enumerate}


	\subsection{UC5: Release: Prepare the publication content}
	\label{sec:uc5}
	\textbf{Trigger:} Asset release is requested
	
	\textbf{Preconditions:} 
	\begin{itemize}
		\item The data is conceptually and formally validated (SRC-AP) and its content is fit to be published
		\item Code freeze is declared, no more changes are foreseen
	\end{itemize}
	
	\textbf{Success guarantee:} The data are available in standard (and where requested additional) forms and formats.
	
	\textbf{Main success scenario:}
	
	\begin{enumerate}
		\item Asset manager requests a release
		\item Data processing officer start the transformation processes from SRC-AP form into 
		\begin{enumerate}
			\item Target forms: SKOS-AP-EU/Act/Core
			\item Target formats: RDF/XML, Turtle, JSON-LD
		\end{enumerate}		
		\item Data processing officer runs the fingerprinting and SHACL validation for structural issues and confirms the transformation went well\footnote{This process is automatic and has the purpose of ensuring the transformation process passed correctly, keeping the data processing officers on the safe side.}.
		\item Data processing officer start the transformation processes from SRC-AP/SKOS-AP-EU into required additional forms and formats such as CAT-XML, XSD, Genericode, Excel/CSV, MarcXML, GeoJSON etc.
		\item Data processing officer places the generated output into the common repository, along with the validation reports, and informs the asset manager and the publication officer		
	\end{enumerate}
	
	\textbf{Extensions:}
	\begin{enumerate}
		\item [3a] The validation reports reveal content related issues:
		\begin{enumerate}
			\item [3a1] The data processing officer identifies and documents the issues and reports them to the quality assessment officer
			\item [3a2] Data authoring officer implements the request for change (UC3)
		\end{enumerate}

		\item [3b] The validation reports reveal data related issues:
		\begin{enumerate}
			\item [3b1] The data processing officer identifies and documents the issues and informs the quality assessment officer
			\item [3b2] The data processing officer fixes the issues due to the transformation process
			\item [3b3] Return to step 2
		\end{enumerate}		
	\end{enumerate}
	

	\subsection{UC6: Publish: Publish a reference data asset}
	\label{sec:uc6}		
	
	\textbf{Trigger:} A publication of selected assets is requested
	
	\textbf{Preconditions:} 
	\begin{enumerate}
		\item The selected assets, validated and converted into all the necessary forms and formats, are available in the common repository
		\item Asset user manual is available in the common repository
		\item Format user manuals are available in the common repository
		\item Asset metadata, both content-related and technical, are available in the common repository
	\end{enumerate}

	\textit{Success guarantee:} The updated assets are accessible on the selected dissemination platforms and the broad public is informed about the new publication

	\textit{Main success scenario:} 
	
	\begin{enumerate}
		\item The scheduled publication due date occurs
		\item The publication officer generates the release notes from the diff-report that summarises what has changed (in more details than the impact assessment).
		\item The publication officer generates the publication summary and impact assessment report (having sections customised for each major stakeholder) that presents an overview of the main content changes and if structural changes are included. 
		\item Asset manager checks the release notes and the impact assessment (that they reflect the change request cases)
		\item Request manager sends the publication summary and impact assessment reports to the stakeholders, to inform them of upcoming changes and collect any pre-publication feedback.
		\item Publication officer runs the packaging process for each asset (parallel to the impact assessment process) resulting in 
		\begin{enumerate}
			\item Generation of additional technical metadata (DCAT, METS, IMMC, etc.)
			\item Generation of packages (ZIP or other) for selected dissemination platforms (Cellar, ODP, Bartoc, Joinup, etc.) that contain all the necessary content, documentation and metadata
		\end{enumerate}
		\item Publication officer tests the integrity/fitness of the generated packages by using the validation mechanisms offered by the dissemination platforms (validators or test dissemination environments)
		\item Request manager receives the implicit acceptance of the impact assessment from the stakeholders (that is, no objections are raised during the established deadline) and informs the publication officer that the assets can be uploaded to the dissemination platform(s). 
		\item Publication officer publishes the packages to the dissemination platform, tests that the assets are accessible and informs the asset manager that publication is completed with success.
		\item Request manager informs the broad public (including stakeholders) that the publication is complete.
		
	\end{enumerate}
	
	\subsection{UC7: Publish: Publish a model asset}
	\label{sec:uc7}
		
	\textbf{Trigger:} A publication of selected assets is requested
	
	\textbf{Preconditions:} 
	\begin{itemize}
		\item The selected assets, which were approved and converted into the standard forms and formats, are available in the common repository
		\item Asset user manual is available in the common repository
		\item Formats/representation user manuals are available in the common repository
		\item Asset metadata, both content-related and technical in the common repository
	\end{itemize}
	
	\textbf{Success guarantee:} The assets are accessible to broad public on the selected dissemination platforms
	
	\textbf{Main success scenario:} 
	\begin{enumerate}
		\item The scheduled publication due date occurs
		\item The publication officer generates automatically the release notes, which summarises the content of the publication.
		\item Request manager sends the publication summary to inform them of upcoming changes and collect any pre-publication feedback.
		\item Publication officer runs the packaging process for each asset resulting in
		\item Publication officer runs the packaging process for each asset (parallel to the impact assessment process) resulting in 
		\begin{enumerate}
			\item Generation of additional technical metadata (DCAT, METS, IMMC, etc.)
			\item Generation of packages (ZIP or other) for selected dissemination platforms (Cellar, IMMC, ODP, Wikidata, Bartoc, Joinup, etc.) that contain all the necessary content, documentation and metadata
		\end{enumerate}
		\item Publication officer tests the integrity/fitness of the generated packages by using the validation mechanisms offered by the dissemination platforms (validators or test dissemination environments)
		\item Publication officer publishes the packages to the dissemination platform, tests that the assets are accessible and informs the asset manager that publication is completed with success
		\item Request manager informs the broad public (including stakeholders) that the publication is complete
		
	\end{enumerate}
	
	\textbf{Extensions:}
	\begin{enumerate}
		\item [6a] Packages are rejected by the dissemination system:
		\begin{enumerate}
			\item [6a1] The publication officer contacts the support team of the dissemination system and resolves the issue. 
			\item [6a2] In case the generated package is incorrect, the publication officer corrects the generation processes.
		\end{enumerate}
	\end{enumerate}
	
\section{Business architecture}
\label{sec:business-architecture}
	
	This section covers in its extent the business architecture. The focus falls almost entirely on the bottom layer of the business architecture structure (see Figure \ref{fig:business-structure-protopypical}) describing the internal processes, events and roles answering the questions who shall do what and when.
	
	
	We address here both, the current organisation and the new organisation of the asset lifecycle process. First we establish a baseline representing the current setup and, second, how the new processes will look like in the light of digital transformations moving towards goals identified in the motivation structure (Section \ref{sec:motivation-architecture}).
	
	Next we explain the general idea of how to the business architecture is structured, which serves as an interpretation framework for the succeeding diagrams. 	
	
	\subsection{Prototypical business structure}
	
	Following the metaphor of layers presented in the motivation view, we decided to explain the  organisation of business structure in terms of layers as well. Figure \ref{fig:business-structure-protopypical} depicts three layers with the most important elements of the business structure. 
	
%	\begin{wrapfigure}{r}{0.4\textwidth}
%		\includegraphics[width=0.4\textwidth]{images/views/Business view.png}
%		\caption{The prototypical business structure view}
%		\label{fig:business-structure-protopypical}
%	\end{wrapfigure}
	
	\begin{figure}[h]
		\centering
		\includegraphics[width=0.4\textwidth]{images/views/Business view.png}
		\caption{The prototypical business structure view}
		\label{fig:business-structure-protopypical}
	\end{figure} 
	
	The topmost layer accounts for the external players or \textit{actors}, which represent a business entity that is capable of performing behaviour, and \textit{roles}, which represent skills and responsibilities for performing specific behaviour, and to which an actor can be assigned \citep{archimate3.1}. 
	
	The middle layer represents the \textit{services} that are offered by the organisation to the external players. A business service represents explicitly defined behaviour that a business role, business actor, or business collaboration exposes to its environment \citep{archimate3.1}.
	

	
	The lower layers accounts for the internal organisation in terms of \textit{events}, \textit{roles}, \textit{processes} and \textit{objects}. The business process represents a sequence of business behaviours that achieves a specific result such as a defined set of products or business services. The business event represents an organizational state change; while a business object represents a (passive) concept used within a particular business domain.
	

	\subsection{Current asset lifecycle stages}
	\label{sec:lifecycle-current-stages}
	
	The current asset lifecycle process is organised in six stages: \textit{inception (or evolution)}, \textit{implementation}, \textit{pre-release}, \textit{release}, \textit{publication} and \textit{consumption}. Each of the stages represents a business sub-process. Figure \ref{fig:lifecycle-current-stages} depicts the order in which stages flow and indicate that each stage process accesses a data asset, the central artefact in the diagram.
	
	\begin{figure}[h]
		\centering
		\includegraphics[width=0.5\textwidth]{images/business/Lifecycle process only (current).png}
		\caption{The current asset lifecycle stages}
		\label{fig:lifecycle-current-stages}
	\end{figure} 

	The \textit{inception} stage means that a request arrives for creating and publishing a new data asset and it is being dealt with by the team. The \textit{evolution} stage is similar only that the request is one for change of an existent data asset. There is no difference in the way these two requests are being treated and so the stage name is a conflation of the two. This stage also includes negotiating the request back with the client and then finally deciding and planning its implementation and publication. 
	
	The \textit{implementation} stage deals with actually changing, authoring, converting (Excel to XML and back) and verifying the client request. 
	
	\textit{Pre-release} marks that the content has been implemented accordingly and can be validated by a second pair of eyes implementing the \textit{four eyes principle} implemented in SU. This verification and validation is performed by checking the validation reports and by comparing the difference between the current and the previous version of the asset conveyed in a diff report. 
	
	In the \textit{release} stage the validated content is placed in a dedicated location of the common repository which indicated that the content is fit for publication. 
	
	The \textit{publication} stage deals with packaging the content and disseminating it to the selected data disseminators, Cellar being the most important one. During this stage a set of announcements and communications ensure that the main stakeholders and the broad public are aware of the published new version of the asset. 
	
	\textit{Consumption} stage is the one that happens outside the SU borders. It is the clients who use the data and then in the process come up with additional request for either changing existent assets or adding and publishing for new ones. 
	
	\subsection{Actors and roles}
	\label{sec:lifecycle-roles}	
	
	This section describes the identified actors and roles relevant to the asset lifecycle process. Figure \ref{fig:internal-roles} depicts their relations.
		
	\textit{Asset manager} (a mix of \textit{operational and business data steward}) is primarily responsible for data content, context, and associated business rules. This role is characterised by the full responsibility for the asset quality, enforcing policies and data governance processes, and ensuring asset fitness (both content and metadata) to the business needs. In the Standardisation Unit, this role is also responsible for high level interaction with the main stakeholders and important clients.
	
	\textit{Team steering committee} (also known as the team meetings) is a body composed of business, technical and analytical roles whose main purpose is to provide executive and operational guidance validating the business requests and assessing both the data management  and the broader impact, determining the implementation priority, and promoting data governance and standardisation practices.
	
%	\begin{figure}[h]
%		\centering
%		\includegraphics[width=0.47\textwidth]{images/business/Internal Business Actors.png}
%		\caption{The actors in metadata and reference data sector}
%		\label{fig:actors-team}
%	\end{figure} 
	
	\begin{figure}[h]
		\centering
		\includegraphics[width=0.7\textwidth]{images/business/Internal Roles.png}
		\caption{The internal roles in metadata and reference data sector}
		\label{fig:internal-roles}
	\end{figure}
	
	\textit{Request manager} is the interface with the client collecting the change requests, assessing the business needs and translating them into data management requirements all being summarised and documented case by case. 
	
	\textit{Data authoring officer} is responsible for editing data in a content management system implementing the cases prepared by the request manager.
	Quality assurance officers validate that the content implementation is correct from a technical and from a business point of view. 
	
	\textit{Data processing officer} is a technical role that is responsible for preparing the assets for publication. The responsibilities include but are not limited to data storage, manipulation, automatic transformation, and generation of validation and assessment reports. 
	
	\textit{Publication officer} is a technical role responsible for packaging and disseminating the assets to the specialised platforms
	
	\textit{Stakeholder steering committee} is a body representing the main clients and stakeholders ensuring data and models harmonisation, alignment of the data management practices and adoption of international standards.
	
	\textit{Client} (\textit{change requester} and \textit{data user}) is a generic external role, who on one side consumes the data and services provided by the Standardisation Unit and on the other side demands publication of new assets or modification of the existing ones. 
	
	\textit{Data disseminator} is an external role providing the Standardisation Unit with reliable data dissemination capabilities, which are meant to make the assets available for the clients.

	The external roles and stakeholders have already been addressed in the motivations structure depicted in Figure \ref{fig:stakehodlers-roles}. Each of these roles has a correspondent element in the business model and will be employed accordingly.

	\subsection{Current asset lifecycle overview}
	\label{sec:lifecycle-current}
	
	This section assembles the asset lifecycle process stages and the main internal roles together in an overview diagram depicted in Figure \ref{fig:lifecycle-current}. It indicated what roles are assigned to which processes, along with cyclical depiction of the process sequence. Next we comment on the involvement of each role in the asset lifecycle process. All the lifecycle stages are internal to teh SU except for the last one, consumption, which takes place at the client premisses.
	
	\begin{figure}[h]
		\centering
		\includegraphics[width=1.05\textwidth]{images/business/Lifecycle (current).png}
		\caption{The current asset lifecycle stages and roles}
		\label{fig:lifecycle-current}
	\end{figure} 
	
	In the inception/evolution stage, the request manager is responsible for creating, documenting and ensuring descriptive completeness for requests arriving from clients to change existent assess or to create of new ones. These requests are managed as individual or, sometimes, interdependent cases. This role serves as the primary interface with the third parties. For this reason, in the publication stage, this role communicates with the stakeholders and broad public about the asset changes when it is published.
	
	The asset manager is in charge of analysing and summarising the request case in the inception/evolution stage. This role intervenes in the pre-release stage to acknowledge that the case has been implemented and the asset is fit for publication; and in the publication stage to check the impact assessment and the release notes before they are used in the communication with external parties.
	
	The team steering committee is involved in the initial stage only. After a new change request case is created, the team steering committee decides whether the case shall be further processed; and if so, then the decision is about when and how.
	
	The data authoring officer is responsible for the case implementation and, in the pre-release stage, verifying and validating its own work as the ``first pair of eyes'' (of the ``two pairs of eyes'' principle). The ``second pair of eyes'' verifying and validating the case implementation is enacted vy the quality assurance officer in the same pre-release stage.
	
	Once the case is marked as fit for publication, in the release stage, it is placed automatically or by intervention of the data processing officer in a region of the common repository tagged for ``release''. If are any technical issues are encountered, the the data processing officers intervenes and fixes them.
	
	
	In the publication stage, the publication officer generates the release artefacts, the release notes, packages the assets and disseminates them to the dissemination partners. External steering committees, such as IMMC metadata sub-committee, GIL EuroVoc an others are asked for final feedback two weeks in advance before the final dissemination.
	
	Next, we turn to discuss the asset lifecycle stages in more detail as elicited from the technical and business teams of the SU. These descriptions are not covering the ultimate details of the process but aim at describing the important building blocks of the current stages. 
	
	\subsection{Current inception and evolution stage}
	\label{sec:inception-current}
	
	\begin{figure}[h]
		\centering
		\includegraphics[width=.8\textwidth]{images/business/current/InceptionEvolution.png}
		\caption{The current process for the inception and evolution stage}
		\label{fig:evolution-current}
	\end{figure}		
	
	The process, depicted in Figure \ref{fig:evolution-current}, starts when a request from a client arrives to either update a data asset or create a new one. This case, after being recorded, analysed and summarised, is asses by the asset manager and then discussed in the team meeting. 
	
	
	In case the case is accepted, then it is scheduled for implementation. Otherwise the case enters a so called negotiation stage, due to one of two things being the case: either the case is incomplete and more information is required from the client, either the case is unacceptable and a rejection is provided to the client with an explanation why or eventually with counter proposals. 
	
	The client communication is mainly carried out by email or telephone conversations, whereas the cases are managed using Jira ticket management system \citep{jira}. 
	
	The process ends when the case is rejected or when it is scheduled for implementation. 

	
	\subsection{Current implementation stage}
	\label{sec:implementation-current}
		
	Figure \ref{fig:implementation-current} depicts the current implementation process. It starts, when the case is queued for implementation. The data authoring officer modifies the asset content according to the instructions provided in the case description. The editing takes place in an Microsoft Excel \citep{excel} workbook which represents an interface to the asset content. Excel is the main editing tool. Once the changes are complete, the workbook is committed into SVN repository \citep{svn}, which triggers and automatic conversion of the Excel workbook into an XML form \cite{xml11-spec}. The XML is considered the primary asset source (structured with CAT XSD scheme). It is further converted back into Excel form, this way entering a conversion loop which also serves as validation mechanism ($XML \rightarrow Excel \rightarrow XML \rightarrow Excel$).
	
	\begin{figure}[h]
		\centering
		\includegraphics[width=.9\textwidth]{images/business/current/Implementation.png}
		\caption{The current process for the implementation stage}
		\label{fig:implementation-current}
	\end{figure}
	
	Once the asset is converted into XML form, it becomes possible through a set of Perl scripts and XSLT style-sheets \citep{xslt3-Kay} to automatically generate assessment artefacts such as the diff report and schema validation report. The diff report indicates what changes have been done to the content between the previous and the latest version, while the validation report contains violations, if any, of XML structural constraints.
	
	The editor then verifies the diff report to ensure the case implementation completeness and that the asset can be tagged for pre-release. Otherwise the content is being edited again. 
	
	The process end with the asset being marked for pre-release, which means that the case implementation is complete.
	
	\subsection{Current pre-release stage}
	\label{sec:pre-release-current}	

	The pre-release stage is depicted in Figure \ref{fig:pre-release-current}. Once the case is marked as implemented, and the assessment artefacts were generated after the conversion into XML form, the second verification and validation can be performed by the quality assurance officer. 
	
	\begin{figure}[h]
		\centering
		\includegraphics[width=.65\textwidth]{images/business/current/Pre-release.png}
		\caption{The current process for the pre-release stage}
		\label{fig:pre-release-current}
	\end{figure}

	In case issues are identified in the case implementation, the quality assurance officer created a new request for changes and the case returns to the implementation stage, otherwise the case implementation is validated and the asset is marked as ready for release. 
	
	This is process where mostly manual steps are taken. Some minor content transformation can take place such as RDF content prettyification and an additional validation of record identifiers in the XML file. These these operations, however, are merely technicalities and do no have business relevance. Therefore they are omitted in the process diagram from Figure \ref{fig:pre-release-current}. 

	\subsection{Current release stage}
	\label{sec:release-current}
	
	\begin{figure}[h]
		\centering
		\includegraphics[width=.6\textwidth]{images/business/current/Release.png}
		\caption{The current process for the release stage}
		\label{fig:release-current}
	\end{figure}

	The release stage is a symbolic step where an asset is simply marked as being in the release stage after it has been verified and validated by four pairs of eyes and confirmed that all the cases had been correctly implemented and that the asset is fit for release in the subsequent publication. This si depicted in Figure \ref{fig:release-current}.

	The release stage is realised by copying the updated version of the asset into a special area of the common repository and marked with the ``release'' tag.

	\subsection{Current publication stage}
	
	The current publication stage is depicted in Figure \ref{fig:publication-current}. It is a wide process that involves almost all the multiples roles. 
	
	\label{sec:publication-current}
		\begin{figure}[h]
		\centering
		\includegraphics[width=1.03\textwidth]{images/business/current/Publication.png}
		\caption{The current process for the publication stage}
		\label{fig:publication-current}
	\end{figure}

	The publication process starts several weeks before the scheduled publication date, when a code freeze is announced and all the pre-selected assets are marked for release. It commences by generating from the CAT-XML source a selection of forms and formats to ease asset consumption by various clients. Mostly XSD, CAT-XML, SKOS-AP-EU forms are generated that are serialised in XML, Turtle and JSON formats. In addition some assets are also prepared in Genericode, Excel/CSV, MarcXML, GeoJSON and other forms. 
	
	Next, the publication notes are prepared together with a few different impact assessments specially prepared for targeted stakeholders. These communication artefacts are checked for correctness and sent to the corresponding stakeholders. After a predefined number of days (usually two weeks) the feedback is collected, if any. Reception of no feedback is considered as a tacit acceptance of the current publication and the process can continue. Seldom, blocking or non-blocking feedback is received which leads to creation of new change requests, and depending on the situation, last minute changes are executed. Or, if the feedback is non-blocking, the intervention is scheduled or the next publication. 
	
	In parallel a packaging process is executed during which, the assets are prepared for dissemination. They are assembled in packages accompanied by their content-related and technical metadata, user manuals, format documentation and, of course, the asset itself expressed in all pre-generated forms and formats, the release artefacts. The packages are verified whether the target dissemination platforms accept them.
	
	Finally the packages are disseminated and the successful publication is announced to the broad public. The dissemination is done primarily through Cellar\citep{cdm-francesconi2015ontology} although other dissemination channels are also employed, among which Wikidata\citep{vrandevcic2014wikidata}, Publications Office Open Data Portal (ODP), Bartoc \citep{ledl2016describing}, JoinUp platform \citep{hillenius2013free}.
	
	This section completes the detailed presentation of the asset lifecycle process as it is today. 
	
	The digital transformation currently undertaken by SU management, that is in part targeted by this architecture, has impact on the asset lifecycle business, application and technical architectures. The next sections will describe how the new asset lifecycle architecture is envisaged. 
		
	\subsection{New asset lifecycle overview}
	\label{sec:lifecycle-new}	
	
	The structure of the new asset lifecycle process is depicted in Figure \ref{fig:lifecycle-new}. In this new architecture, we aim to propose incremental changes, as to cause least disruptions to the team and the ongoing operations. 

	\begin{figure}[h]
		\centering
		\includegraphics[width=1.05\textwidth]{images/business/Lifecycle (new).png}
		\caption{The current asset lifecycle stages and roles}
		\label{fig:lifecycle-new}
	\end{figure}
	
	This process is similar, in its structure and stage names, to the current one: six stages circularly linked. The only change is the replacement of pre-release stage a stage called validation. There are however significant changes in the structure of three stages: implementation, validation and release, while the inception/evolution stage is identical to the current one, and the publication stage are is almost entirely unchanged. These changes are addressed in detail in the next sections.
		
	There are no new roles in the new process, but there is a difference in the role allocation and involvement at different stages of the process as compared to the current one. 
	
	As mentioned above, the implementation/evolution stage is identical to the current one. So after the case has been registered and processed the implementation is scheduled. How it happens in the new workflow is presented in the next section. 
			
	\subsection{New implementation stage}
	\label{sec:implementation-new}

	The new implementation process is depicted in Figure \ref{fig:validation-new}. The main digital transformation at this stage is adoption of RDF asset source, and the switch from Microsoft Excel as the content editor to VocBench3 \citep{stellato2017towards,stellatovocbench} semantic web editor. 
	
	\begin{figure}[h]
		\centering
		\includegraphics[width=1.05\textwidth]{images/business/new/Implementation.png}
		\caption{The new process for the implementation stage}
		\label{fig:implementation-new}
	\end{figure} 

	The overall process looks very similar to the current one, but the employed technology for the source editing: VocBench3 and SKOS model(s) \citep{skos-spec}, render an entirely different editor experience when modifying the source content to implement the request case.
	
	When the editing is complete, the content is exported from VocBench into the common repository and a set of asset assessment artefacts are generated. These artefacts are: the asset diff report, the fingerprint report and the validation report (based on SHACL \citep{shacl-spec} validation rules). These reports are no longer implemented based on the XSLT technology \citep{xslt3-Kay}, because the underlying source is the new RDF representation \citep{rdf11} and no longer (CAT-)XML. More technical details are addressed in the application architecture in Section \ref{sec:implementation-application}.
	
	The editor (data authoring officer) then verifies that the case is implemented correctly, and if so a validation, by the quality assurance office is scheduled, otherwise, if some issues are spotted more editing is performed in VocBench3.
		
	
	\subsection{New validation stage}
	\label{sec:validation-new}	
	
	The validation stage is new in the asset lifecycle process. Here the correctness assessment is separated into two steps: verification and validation. This is a continuation of the current conception of how the case implementation is being assessed. The verification is primarily focused on the business aspects of the case implementation; whereas, the validation, deals with the technical and more formal aspects of the content structure and consistency. As the new technology allows for semantic accounts, then validation extends to deal with semantics as well. 	 

	\begin{figure}[h]
		\centering
		\includegraphics[width=.892\textwidth]{images/business/new/Validation.png}
		\caption{The new process for the validation stage}
		\label{fig:validation-new}
	\end{figure}

	In case errors are detected in the implementation, the quality assurance officer issues a  change request and the case goes back into the implementation stage. Otherwise, the case implementation is considered correct and from now on a release can be scheduled. This is performed by the asset manager. 
		
	\subsection{New release stage}
	\label{sec:release-new}
	
	The new release stage, depicted in Figure \ref{fig:release-new}, differs considerably from the current one. This stage is no longer one of simply marking the asset implementation as fit for publication but deals with all the data transformations and preparation of artefacts necessary in the publication stage.
	
	\begin{figure}[h]
		\centering
		\includegraphics[width=1.05\textwidth]{images/business/new/Release.png}
		\caption{The new process for the release stage}
		\label{fig:release-new}
	\end{figure}

	The release starts when all the change request cases, in the scope of the scheduled publication, have been implemented and validated. Then, the primary release artefacts are generated from the RDF source expressed in SRC-AP form \citep{src-ap-vb3}. The main artefacts are the source in SKOS-AP-EU form \citep{skos-ap-eu}, a special extension of the SKOS-AP-EU which is necessary for publishing the content in Cellar because of the backwards compatibility issues (called SKOS-AP-EU-Act), and the SKOS core \citep{skos-spec} representation. These artefacts are also generated in RDF/XML \citep{rdf-xml-Schreiber:14:RXS,rdf-xml-Beckett:04:RSS}, Turtle \citep{turtle-Carothers:14:RT} and JSON-LD \citep{sporny2014json,spornyjson} formats. 
	
	The transformation process is accompanied by a set of validation processes meant to safeguard the and ensure that the transformations are executed correctly. The validation reports are checked by the data processing officer. In case process related issues are detected, then most likely the processes contain bugs and need to be fixed. In case content related issues is spotted then a new change request is created and the process goes back to the implementation stage. 
	
	After the primary release artefacts are created, then secondary release artefacts are created. All of them are non-RDF forms and formats that currently are produced for clients and stakeholders and must continue so. An automated validation of the generated assets, to the extent possible, secures and ensures quality of the output. Like in the case of the primary artefacts, if the errors are detected then depending on their nature, either the data transformation process must be updated or a content change request is issued and the process returns back to the implementation stage. 
	
	The transformation technology employed here needs to provide ETL\footnote{Extract, transform, load (ETL) is the general procedure of copying data from one or more sources into a destination system which represents the data differently from the source(s) or in a different context than the source(s).}/ELT\footnote{Extract, load, transform (ELT) is a variant of ETL where the extracted data is loaded into the target system first.}-like capabilities for both: RDF and for data formats. An extended discussion about the application capabilities is covered in Section \ref{sec:release-application}.
	
	At the end of this stage, all the important asset transformations and data conversions must be complete and the necessary forms and formats consumed by target clients shall be available for publication. Finally the assets are copied into the special place of the common repository available to the publication process. 
	
	\subsection{New publication stage}
	\label{sec:publication-new}
	
	The new publication stage process is depicted in Figure \ref{fig:publication-new}. This process is very similar to the one currently performed described in Section \ref{sec:publication-current}.
	
	\begin{figure}[h]
		\centering
		\includegraphics[width=1.05\textwidth]{images/business/new/Publication.png}
		\caption{The new process for the release stage}
		\label{fig:publication-new}
	\end{figure}

	The main difference in this process is the starting point. If currently it starts by transforming and converting the asset source into all forms and formats demanded by the clients, then the new process starts from the premise that the entire publication content has already been generated. 
	
	The new process stats by generation of the release notes and the sett of impact assessment reports destined for different stakeholders. This communication artefacts are verified by the asset manager and then the request manager spreads the news to the stakeholders and the broad public about the upcoming publication. The stakeholders may have a veto or comment on the current publication, just like it currently foreseen.   
	
	What is left, at this stage, from the technical point of view, is packaging and distributing of the publication packages to the data disseminators. The process ends when the latest version of published assets are accessible via the selected data disseminators and the broad public is informed about this. 
	
	This section finishes the description of the business architecture. Next is presented the the application architecture accounting for necessary services and capabilities that must be in place in order to serve the current business processes along with an account of what components realize such services. 
	